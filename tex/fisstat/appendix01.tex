\section{Cálculo multivariável}

\epigraph{\justifying A Filosofia está escrita neste grande livro, o Universo,
que está permanentemente aberto e ao alcance do nosso olhar. Mas o livro não
pode ser compreendido sem antes aprendermos a linguagem e os caracteres em que
está escrito. A linguagem é a Matemática, e os caracteres são triângulos, 
círculos e outras figuras geométricas, sem as quais é humanamente impossível
compreender uma única palavra. Sem estes, ficamos perambulando dentro de um
escuro labirinto.} {\emph{Galileu Galilei}, Il Saggiatore (1623)}

\noindent As técnicas do estudo do comportamento local de funções podem ser
facilmente extendidas para funções cujo domínio, ao invés de intervalos em
$\mathbb R$, são bolas no $\mathbb R^n$. Muitos resultados tornam-se mais claros
nessa linguagem e portanto devemos dar uma boa atenção para ela. Nossos objetos
de estudo serão funções $f:\mathbb R^n\rightarrow\mathbb R$.

\subsection{A derivada direcional}

Começamos com a seguinte questão: como definir a noção de derivada, ou taxa de
variação, para uma função que depende de mais de um parâmetro? Lembrando que
quando haviamos apenas uma variável, a derivada de uma função $g:\mathbb R
\rightarrow\mathbb R$ era definida como
$$g'(t)=\lim\limits_{h\rightarrow0}\frac{g(t+h)-g(t)}{h}.$$
Para conseguimos utilizar essa mesma ideia em várias variáveis, considere alguma
trajetória no espaço $\mathbb R^n$,
$$\gamma(t)=(\gamma^1(t),\gamma^2(t),\dots,\gamma^n(t)).$$
A composição $f\circ\gamma$, portanto, é uma função em que podemos definir a
derivada tradicional:
$$(f\circ\gamma)'(t)=
\lim\limits_{h\rightarrow0}\frac{f(\gamma(t+h))-f(\gamma(t))}{h},$$
chamada de \emph{derivada direcional} ou \emph{derivada total} de $f$ em $\gamma
(t)$ ao longo de $\gamma'(t)$. No entanto, gostariamos de definir uma noção de
derivada independente da trajetória utilizada, e a maneira que escrevemos acima
parece sugerir que não há uma forma trivial de separar qual parte da derivada é
devido à função $f$, e qual é devido a nossa escolha de trajetória $\gamma$.
Somando vários zeros na expressão no interior do limite, no entanto:
\begin{align*}
    \frac{f(\gamma(t+h))-f(\gamma(t))}{h}
    &=\frac{f(\gamma^1(t+h),\gamma^2(t+h),\dots,\gamma^n(t+h))
    -f(\gamma^1(t),\gamma^2(t),\dots,\gamma^n(t))}{h}\\
    &=\frac{f(\gamma^1(t+h),\gamma^2(t+h),\dots,\gamma^n(t+h))
    -f(\gamma^1(t),\gamma^2(t+h),\dots,\gamma^n(t+h))}{h}\\
    &+\frac{f(\gamma^1(t),\gamma^2(t+h),\dots,\gamma^n(t+h))
    -f(\gamma^1(t),\gamma^2(t),\dots,\gamma^n(t+h))}{h}\\
    &+\cdots\\
    &+\frac{f(\gamma^1(t),\gamma^2(t),\dots,\gamma^n(t+h))
    -f(\gamma^1(t),\gamma^2(t),\dots,\gamma^n(t))}{h}
\end{align*}
e podemos usar uma lógica análoga à regra da cadeia para escrever o limite de
cada uma destas frações, afirmando que
$$\lim\limits_{h\rightarrow0}\frac{f(\dots,\gamma^i(t+h),\dots)
-f(\dots,\gamma^i(t),\dots)}{h}=f_{,i}(\gamma(t))\gamma^i{'}(t)$$
onde
$$f_{,i}(x)\equiv\lim\limits_{h\rightarrow0}
\frac{f(\dots,x^i+h,\dots)-f(\dots,x^i,\dots)}{h}$$
é chamada a \emph{derivada parcial} de $f$ na direção $x^i$. Essa é calculada
assim como uma derivada usual, somente na variável de interesse enquanto as
outras são consideradas coeficientes constantes. Outras notações comuns para
essa grandeza são
$$\partial_i f\quad\text{ou}\quad\pd{f}{x^i}{}.$$
Nossa expressão para a derivada total então fica
$$(f\circ\gamma)'(t)=f_{,1}(\gamma(t))\gamma^{1}{'}(t)
+\cdots+f_{,n}(\gamma(t))\gamma^{n}{'}(t).$$
Isso nos mostra que a derivada direcional depende apenas do valor e da primeira
derivada da trajetória. ou deixando a trajetória em questão implícita,
escrevemos uma variação genérica do valor de $f$ em termos das variações dos
valores das coordenadas como
$$df=f_{,1}dx^1+\cdots+f_{,n}dx^n.$$
Podemos definir também derivadas de maior ordem encadeando esse procedimento
recursivamente
$$f_{,i_1i_2\dots i_N}(x)=(f_{,i_2\dots i_N})_{,i_1}(x),$$
também com notações alternativas
$$\partial_{i_1\dots i_N} f\quad\text{ou}
\quad\frac{\partial^N f}{\partial x^{i_1}\cdots\partial x^{i_N}}.$$
Quando essas derivadas são contínuas, elas não dependem da ordem em que as
tomamos, valendo que
$$f_{,i_1\dots i_Nj_1\dots j_M}(x)=f_{,j_1\dots j_Mi_1\dots i_N}(x).$$

\subsection{Funções Implícitas}

Muitas vezes nos encontramos em situações onde ao invés de estar interessados em
funções, queremos entender o comportamento de variáveis relacionadas por
equações. Considere o caso em que temos $n+m$ variáveis que vamos denotar por
$x^1,\dots,x^n,y^1,\dots,y^m$, sujeitas a $m$ equações
\begin{align*}
    f_1(x^1,\dots,x^n,y^1,\dots,y^m)&=0,\\
    f_2(x^1,\dots,x^n,y^1,\dots,y^m)&=0,\\
    \cdots&\\
    f_m(x^1,\dots,x^n,y^1,\dots,y^m)&=0.
\end{align*}

Definimos neste caso o \emph{Jacobiano} das funções $f_1,\dots,f_m$ com relação
as variáveis $y^1,\dots,y^m$ como a matriz $J(x^1,\dots,x^n,y^1,\dots,y^m)$
cujas entradas são dadas por 
$$J=\left(f_{i,y^j}\right).$$
Assim, vale o seguinte 
\begin{theorem}[Função implícita]
    Dado $(x_0,y_0)=(x^1_0,\dots,x^n_0,y^1_0,\dots,y^m_0)$ tal que $f_1(x_0,y_0)
    =\cdots=f_m(x_0,y_0)=0$, se todas as componentes de $J$ são contínuas numa
    vizinhança de $(x_0,y_0)$ e $\det J(x_0,y_0)\neq0$, então existem funções
    únicas $\varphi^1(x),\dots,\varphi^m(x)$ tal que $\varphi^1(x_0)=y^1_0,
    \dots, \varphi^m(x_0)=y^m_0$ e $f_1(x,\varphi(x))=\cdots=f_m(x,\varphi(x))
    =0$ numa vizinhança de $x_0$.
\end{theorem}

Note que a separação entre as variáveis independentes $x$ e as dependentes $y$ é
arbitrária, e separações diferentes levam a construções de funções diferentes, e
que somente algumas podem satisfazer as hipóteses do teorema ao que se refere o
Jacobiano enquanto outras não.

Como rascunho de demonstração, considere que as variações dos $y^k$ devem ser
tais que, junto das variações dos $x^j$, cancelam qualquer tipo de variação nos
$f_i$:
\begin{align*}
    df_1&=\quad f_{1,x^1}dx^1+\cdots+
    f_{1,x^n}dx^n+f_{1,y^1}dy^1+\cdots+f_{1,y^m}dy^m=0\\
    df_2&=\quad f_{2,x^1}dx^1+\cdots+
    f_{2,x^n}dx^n+f_{2,y^1}dy^1+\cdots+f_{2,y^m}dy^m=0\\
    &\quad\cdots\\
    df_m&=\quad f_{m,x^1}dx^1+\cdots+
    f_{m,x^n}dx^n+f_{m,y^1}dy^1+\cdots+f_{m,y^m}dy^m=0
\end{align*}
Temos portanto o seguinte sistema linear de $m$ equações:
\begin{align*}
    J_{11}dy^1+\cdots+J_{1m}dy^m&=-f_{1,x^1}dx^1-\cdots-f_{1,x^n}dx^n\\
    J_{21}dy^1+\cdots+J_{2m}dy^m&=-f_{2,x^1}dx^1-\cdots-f_{2,x^n}dx^n\\
    \cdots&\\
    J_{m1}dy^1+\cdots+J_{mm}dy^m&=-f_{m,x^1}dx^1-\cdots-f_{m,x^n}dx^n
\end{align*}
que, dado a condição $\det J\neq0$ tem solução dada pela matriz inversa:
\begin{align*}
    dy^1&=-(J^{-1})^{11}\left(f_{1,x^1}dx^1+\cdots+f_{1,x^n}dx^n\right)-
    \cdots-(J^{-1})^{1m}\left(f_{m,x^1}dx^1+\cdots+f_{m,x^n}dx^n\right)\\
    dy^2&=-(J^{-1})^{21}\left(f_{1,x^1}dx^1+\cdots+f_{1,x^n}dx^n\right)-
    \cdots-(J^{-1})^{2m}\left(f_{m,x^1}dx^1+\cdots+f_{m,x^n}dx^n\right)\\
    \cdots&\\
    dy^m&=-(J^{-1})^{m1}\left(f_{1,x^1}dx^1+\cdots+f_{1,x^n}dx^n\right)-
    \cdots-(J^{-1})^{mm}\left(f_{m,x^1}dx^1+\cdots+f_{m,x^n}dx^n\right)\\
\end{align*}
Podemos então coletar os termos de $dx^j$ em cada uma destas equações e escrever
um sistema de $mn$ equações diferenciais
$$dy^i=\varphi^i_{,1}dx^1+\cdots+\varphi^i_{,n}dx^n \Rightarrow
\varphi^i_{,j}=-(J^{-1})^{i1}f_{1,x^j}-\cdots-(J^{-1})^{im}f_{m,x^j}.$$
Esse sistema de equações possui solução única dada a condição inicial $\varphi
(x_0)=y_0$, e vale na vizinhança de $x_0$ onde $\det J\neq0$.

Com esse maquinário podemos então definir as seguintes grandezas para todo $y^i$
e todo $x^j$:
$$\pd{y^i}{x^j}{x^1,\dots,x^{j-1},x^{j+1},\dots,x^n}\equiv\phi^i_{,j},$$
que é lido como a \emph{derivada parcial de $y^i$ com relação a $x^j$ a $x^1,
\dots,x^{j-1},x^{j+1} ,\dots,x^n$ constantes}. As variáveis constantes precisam
ser explicitadas para determinar qual foi a construção realizada com o teorema
da função implícita, mas pode ser omitida em casos em que a decomposição
utilizada fica clara do contexto. No caso $n=1$ denotamos essa grandeza
simplesmente com a notação de Leibniz usual, $\frac{dy^i}{dx}$. Isso nos permite
escrever a variação de cada $y^i$ como
$$dy^i=\pd{y^i}{x^1}{x^2,\dots,x^n}dx^1+\pd{y^i}{x^2}{x^1,x^3,\dots,x^n}dx^2
+\cdots+\pd{y^i}{x^n}{x^1,\dots,x^{n-1}}dx^n.$$
Derivadas de maior ordem podem também ser construidas com seu encadeamento,
desde que a decomposição utilizada no teorema da função implícita seja a mesma
para todas elas, e tem a mesma propriedade de simetria de permutação da derivada
parcial usual. No caso em que as variáveis dependentes já são funções explícitas
das variáveis dependentes, a solução do teorema da função implícita são as
funções em questão, e as derivadas parciais a variáveis constantes são apenas as
derivadas parciais das funções originais.

Por fim, uma trajetória $\gamma(t)$ nas variáveis independendes extende-se
naturalmente para as variáveis dependentes atrvés da função $\varphi$ como 
$(\gamma(t), \varphi\circ\gamma(t))$.

\subsection{Transformação de Legendre}

Suponha que estamos estudando a variável $y$ na relação $f(q,x,y)=0$ e estamos
particularmente interessados no valor da derivada 
$$p=\pd{y}{q}{x}$$
em vez do valor de $q$. Inspirando-se na seguinte manipulação:
$$dy=pdq+\pd{y}{x}{q}dx\Rightarrow d(y-pq)=-qdp+\pd{y}{x}{q}dx,$$
que sugere que a variável $z=y-pq$ é uma função implícita de $p$ sem ser função
implícita de $q$, vamos utilizar o teorema da função implícita no sistema de
equações
\begin{align*}
    f(q,x,z+pq)&=0\\
    f_{,y}(q,x,z+pq)p+f_{,q}(q,x,z+pq)&=0
\end{align*}
deixando $z$ e $q$ como função de $p$ e $x$. A conclusão que chegaremos é de que
$$dz=-qdp+\pd{y}{x}{q}dx,$$
ou seja que de fato
$$\pd{z}{p}{x}=-q,\quad\pd{z}{x}{p}=\pd{y}{x}{q}$$
e a condição no Jacobiano para que isso seja possível além da que já é tomada
como pressuposto quando tornamos $y$ função implícita torna-se simplesmente
$$\pd{p}{q}{x}\neq0.$$

A variável $z$ recebe o nome de \emph{transformação de Legendre} de $y$ em
relação a $q$. A construção acima pode ser facilmente extendida para uma relação
de várias variáveis $f(q^1,\dots,q^m,x^1,\dots,x^n,y)=0$, com
$$z=y-p_1q^1-\cdots-p_mq^m,$$
e para todos $i$ vale que
$$\pd{z}{p_i}{p_1,\dots,p_{i-1},p_{i+1},\dots,p_m,x^1,\dots,x^n}=-q^i,$$
$$\pd{z}{x^i}{p_1,\dots,p_m,x^1,\dots,x^{i-1},x^{i+1},\dots,x^n}=
\pd{y}{x^i}{q^1,\dots,q^m,x^1,\dots,x^{i-1},x^{i+1},\dots,x^n},$$
sob a condição no Jacobiano de $p$ em relação a $q$
$$\det\left(\pd{p_i}{q^j}{q^1,\dots,q^{j-1},q^{j+1},
\dots,q^m,x^1,\dots,x^n}\right)\neq0$$
A construção também claramente vale quando $n=0$ e todas as variáveis
independentes de uma relação sofrem a transformação ao mesmo tempo. Além disso,
tudo aqui vale trivialmente quando a variável dependente em questão é uma função
explícita das variáveis dependentes.

\subsection{Problemas}

\begin{enumerate}
    \item
        Considere a função $f(x,y) = 3x^2y-2x$ e a trajetória $\gamma(t)=
        (\cos t,\sin t)$.
        \begin{enumerate}
            \item
                Usando a definição de derivada direcional, encontre a derivada
                de $f$ em $\gamma(0)$ ao longo de $\gamma'(0)$
                \answer{$(f\circ\gamma)'(0)=3$}
            \item
                Utilize a decomposição em derivadas parciais e a velocidade da
                trajetória para obter o mesmo resultado
                \answer{$f_{,x}=6xy-2,f_{,y}=3x^2,\gamma'(t)=(-\sin t,\cos t)$}
        \end{enumerate}

        \item Encontre todas as derivadas parciais das seguintes funções:
        \begin{enumerate}
            \item 
                $f(x,y,z)=4x^3y^2-e^zy^4+\frac{z^3}{x^2}_4y-x^{16}$
                \answer{
                    $f_{,x}=12x^2y^2-\frac{2z^3}{x^3}-16x^{15},
                    f_{,y}=8x^3y-4e^zy^3+4,f_{,z}=-e^zy^4+\frac{3z^2}{x^2}$
                }
            \item 
                $g(s,t,u)=t^2\log(s+2t)-\log(3u)(s^3+t^2-4u)$
                \answer{
                    $g_{,s}=\frac{t^2}{s+2t}-3s^2\log(3u),
                    g_{,t}=2t\log(s+2t)+\frac{2t^2}{s+2t}-2t\log(3u),
                    g_{,u}=4\log(3u)-\frac{s^3+t^2-4u}{u}$
                }
        \end{enumerate}

    \item
        Dado o sistema de equações
        \begin{align*}
            x^2y+3y+z^3-z&=8\\
            2x+2y+\cos xz&=7
        \end{align*}
        que possui solução $(x,y,z)=(1,2,0)$, encontre um par de variáveis que
        podem ser escritas como função da outra em torno deste ponto e escreva
        as equações diferenciais associadas. Calcule o valor das derivadas no
        ponto dado.
        \answer{
            $\frac{dy}{dx}=-\frac{2x^2y\sin xz+(2-z\sin xz)(3z^2-1)}
            {(x^2+3)x\sin xz+2(3z^2-1)}=-1,
            \frac{dz}{dx}=-\frac{4xy-(x^2+3)(2-z\sin xz)}
            {(x^2+3)x\sin xz+2(3z^2-1)}=0$
        }

    \item
        A esfera unitária no $\mathbb R^3$ é definida pela equação $x^2+y^2+z^2
        =1$.
        \begin{enumerate}
            \item
                Encontre $z$ como função de $x,y$ de forma que a solução inclua
                o ponto $(0,0,1)$.
                \answer{$z=\sqrt{1-x^2-y^2}$}
            \item
                Utilize o teorema da função implícita em torno de $(0,0,1)$
                utilizando $x,y$ como variáveis independentes e $z$ como
                dependente para escrever um par de equações diferenciais. Mostre
                que sua resposta do item anterior satisfaz as equações.
                \answer{$\pd{z}{x}{y}=-\frac{x}{z},\pd{z}{y}{x}=-\frac{y}{z}$}
        \end{enumerate}

    \item
        Mostre que para uma relação $f(x,y,z)=0$ vale que
        \begin{align*}
            \pd{x}{y}{z}\pd{y}{x}{z}&=1\\
            \pd{x}{y}{z}\pd{y}{z}{x}\pd{z}{x}{y}&=-1
        \end{align*}

    \item
        Mostre que para um par de relações $f(x,y,z,w)=g(x,y,z,w)=0$ vale que
        \begin{align*}
            \pd{w}{z}{x}&=\pd{w}{y}{x}\pd{y}{z}{x}\\
            \pd{w}{x}{z}&=\pd{w}{x}{y}+\pd{w}{y}{x}\pd{y}{x}{z}
        \end{align*}

    \item
        Encontre a transformada de Legendre das seguintes funções
        \begin{enumerate}
            \item
                $f(t)=e^t$
                \answer{$g(s)=s(\log s-1)$}
            \item
                $x(q)=q^2$
                \answer{$y(p)=\frac{1}{4}p^2$}
        \end{enumerate}

    \item
        Suponha que uma variável $a$ dependa de $q$ e $x$ com o diferencial
        $$da=pdq+ydx.$$
        Mostre que
        $$\pd{a}{x}{p}=y-p\pd{y}{p}{x}$$

    \item
        * Ao final do século XVIII foi desenvolvida a mecânica analítica por
        Giuseppe Lagrangia. A abordagem consiste em resumir todos os efeitos da
        mecânica clássica através de uma função $L(q,v,t)$ que caracteriza o
        sistema em questão, dependendo da posição $q$, velocidade $v$ e do tempo
        $t$. As equações de movimento para a trajetória $\gamma_L(t)=
        (\gamma_L^q(t),\gamma_L^v(t),t)$ do sistema então são geradas pelo par
        de equações diferenciais
        \begin{align*}
            \gamma_L^q{'}(t)&=\gamma_L^v(t)\\
            \left(L_{,v}\circ\gamma_L\right)'(t)
            &=\left(L_{,q}\circ\gamma_L\right)(t)
        \end{align*}
        e condição inicial $\gamma_L(0)=(q_0,v_0,0)$. Começando com um exemplo,
        \begin{enumerate}
            \item
                Mostre que se $L(q,v,t)=\frac{1}{2}mv^2-\frac{1}{2}kq^2$
                obtemos a equação de movimento para a distensão $q$ de uma massa
                $m$ presa a uma mola de constante elástica $k$
        \end{enumerate}
        Porém o verdadeiro poder da abordagem que estamos considerando é a
        possibilidade de escrever equações de movimento utilizando sistemas de
        coordenadas alternativos quando os eixos cartesianos geram equações
        inelegantes. Suponha que gostariamos de descrever um sistema utilizando
        uma coordenada $Q$ que está relacionada à coordenada anterior por uma
        relação $f(q,Q,t)=0$.
        \begin{enumerate}
            \setcounter{enumii}{1}
            \item
                Encontre uma relação $g(q,v,Q,V,t)=0$ para uma nova variável $V$
                de forma que, utilizando o teorema da função implícita para
                construir $(q,v)=\varphi(Q,V,t)$, para qualquer trajetória 
                $\Gamma_L(t)=(\Gamma_L^Q(t),\Gamma_L^V(t),t)$
                $$\Gamma_L^Q{'}(t)=\Gamma_L^V(t)\Rightarrow(\varphi^q\circ
                \Gamma_L){'}(t)=(\varphi^v\circ\Gamma_L)(t)$$
                \answer{
                    $g(q,v,Q,V,t)=f_{,q}(q,Q,t)v+f_{,Q}(q,Q,t)V+f_{,t}(q,Q,t)$
                }
            \item
                Definindo
                $$\hat L(Q,V,t)=L(\varphi(Q,V,t),t)$$
                Mostre que se $\Gamma_L(t)$ for tal que 
                \begin{align*}
                    \Gamma_L^Q{'}(t)&=\Gamma_L^V(t)\\
                    \left(\hat L_{,V}\circ\Gamma_L\right)'(t)
                    &=\left(\hat L_{,Q}\circ\Gamma_L\right)(t)
                \end{align*}
                sujeito à condição inicial $(\varphi\circ\Gamma_L)(0)=(q_0,
                v_0)$, então $\gamma_L(t)=(\varphi\circ\Gamma_L,t)$.
        \end{enumerate}

        Outro ponto de vista da mecânica analítica que tem um sabor parecido com
        a de Lagrangia é o de William Hamilton. Aqui vamos caracterizar o
        sistema por uma função $H(q,p,t)$ da posição $q$, momento $p$ e tempo
        $t$. As equações de movimento para a trajetória $\gamma_H(t)=
        (\gamma_H^q(t),\gamma_H^p(t),t)$ agora tornam-se
        \begin{align*}
            \gamma_H^q{'}(t)&=(H_{,p}\circ\gamma_H)(t)\\
            \gamma_H^p{'}(t)&=-(H_{,q}\circ\gamma_H)(t)
        \end{align*}
        e condição inicial $\gamma_H(0)=(q_0,p_0,0)$.
        \begin{enumerate}
            \setcounter{enumii}{3}
            \item
                Mostre que se $H(q,p,t)=\frac{p^2}{2m}+\frac{1}{2}kq^2$ obtemos
                a equação de movimento para a distensão $q$ de uma massa $m$
                presa a uma mola de constante elástica $k$
        \end{enumerate}
        Note a grande simetria entre as duas equações quando comparada com as de
        Lagrangia. Isso nos permite fazer mudanças de coordenadas ainda mais
        esotéricas. Suponha que construimos um par de relações entre $q,p,Q,P,t$
        de forma que
        $$\pd{Q}{q}{p,t}\pd{P}{p}{q,t}-\pd{Q}{p}{q,t}\pd{P}{q}{p,t}=1.$$
        Neste caso a transformação de coordenadas é dita \emph{canônica}.
        \begin{enumerate}
            \setcounter{enumii}{4}
            \item
                Mostre que se uma transformação é canônica,
                $$\pd{q}{Q}{P,t}=\pd{P}{p}{q,t}$$
                e conclua que existe uma função $F(q,P,t)$ tal que $p=F_{,q}$ e
                $Q=F_{,P}$. Esta é chamada de \emph{função geratriz} da
                transformação canônica.
            \item
                Usando o teorema da função implícita para construir $(q,p)=\psi
                (Q,P,t)$ e definindo
                $$\hat H(Q,P,t)=H(\psi(Q,P,t),t)+F_{,t}(\psi^q(Q,P,t),P,t)$$
                Mostre que se $\Gamma_H(t)$ for tal que
                \begin{align*}
                    \Gamma_H^Q{'}(t)&=(\hat H_{,P}\circ\Gamma_H)(t)\\
                    \Gamma_H^P{'}(t)&=-(\hat H_{,Q}\circ\Gamma_H)(t)
                \end{align*}
                sujeito à condição inicial $(\psi\circ\Gamma_H)(0)=(q_0,p_0)$,
                então $\gamma_H(t)=(\psi\circ\Gamma_H)(t)$.
        \end{enumerate}
        Apesar do formalismo de Hamilton parecer mais poderoso que o de
        Lagrangia por podermos misturar posições e momentos, eles são em grande
        parte equivalentes, sendo relacionadas por uma transformação de Legendre
        entre as variáveis $v$ e $p$,
        $$H=-(L-pv).$$
        \begin{enumerate}
            \setcounter{enumii}{6}
            \item
                Caracterize a função geratriz da transformação de coordenadas
                estudada nos itens (b) e (c)
                \answer{$F(q,P,t)=P\varphi(q,t),\quad f(q,\varphi(q,t),t)=0$}
        \end{enumerate}
\end{enumerate}

