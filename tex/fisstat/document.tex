\documentclass[a4paper, 12pt]{article}

\usepackage[colorlinks=false,linktocpage=true]{hyperref}

\usepackage{fancyhdr}
\pagestyle{fancy}
\renewcommand{\headrulewidth}{0.4pt}
\fancyheadoffset{0.5cm}
\setlength{\headheight}{20pt}

\usepackage[margin=2cm]{geometry}
\usepackage{float}

\usepackage{tikz}
\usetikzlibrary{calc}
\usetikzlibrary{decorations.pathmorphing}
\usetikzlibrary{decorations.markings}
\tikzset{
    mid arrow/.style={
        postaction={
            decorate,decoration={
                markings,
                mark=at position .5 with {\arrow[#1]{stealth}}
            }
        }
    },
}
\usetikzlibrary{intersections}
\usetikzlibrary{spath3}
\usetikzlibrary{hobby}

\usepackage[utf8]{inputenc}
\usepackage[T1]{fontenc} 
\usepackage[brazilian]{babel}

\usepackage{amsmath}
\usepackage{amssymb}
\usepackage{amsthm}

\usepackage{etoolbox}
\newcommand\problemanswers{}
\makeatletter
\newcommand{\answer}[1]{
    \listxadd{\problemanswers}{
        \thesection.\theenumi\ifnum\@enumdepth=2\theenumii\fi) \unexpanded{#1}
    }
}
\makeatother

\newcommand{\pd}[3]{
    \ifblank{#3}{}{\left(}
    \frac{\partial #1}{\partial #2}
    \ifblank{#3}{}{\right)_{#3}}
}

\usepackage{ragged2e}
\usepackage{epigraph}
\setlength\epigraphwidth{.8\textwidth}
\usepackage{csquotes}

\newtheorem{theorem}{Teorema}[section]
\newtheorem{lemma}{Lema}
\theoremstyle{definition}
\newtheorem*{definition}{Definição}
\theoremstyle{definition}
\newtheorem{law}{Lei}
\setcounter{law}{-1}

\title{Física Estatística}
\author{Gabriel Golfetti}
\date{}

\begin{document}

\maketitle
\tableofcontents

\section{Introdução}

\epigraph{\justifying A capacidade de reduzir tudo a leis fundamentais simples
não implica na capacidade de partir destas e reconstruir o universo. A hipótese
construcionista se desfaz quando confrontada com o par de dificuldades de escala
e complexidade. A cada camada de complexidade surgem propriedades totalmente
novas. Psicologia não é uma aplicação da biologia, e muito menos a biologia uma
aplicação da química. Agora vemos que o todo torna-se não algo além, mas algo
completamente diferente da soma de suas partes.}{\emph{Philip W. Anderson}, More
is Different: Broken Symmetry and the Nature of the Hierarchical Structure of
Science (1972)}

\noindent A termodinâmica é área da física que lida com \emph{emergência}, ou
seja, propriedades, leis ou fenômenos que ocorrem em escalas macroscópicas -
quando o número de constituintes torna-se grande - que não surgem naturalmente
na escala microscópica - na dinâmica fundamental dos constituintes, apesar de
por definição o sistema macroscópico podendo ser visto como um grande conjunto
de sistemas microscópicos sujeitos a tais leis dinâmicas.

\subsection{Escala}

Com o objetivo de capturar o que acontece quando nossos sistemas tornam-se
grandes e macroscópicos, introduzimos a noção de \emph{transformação de escala}
para os nossos sistemas de interesse, onde caracterizamos o comportamento de uma
grandeza física $X$ quando alteramos o tamanho por um fator $\lambda$ com uma 
\emph{lei de escala} dada por uma função positiva e monotônica $f$. Escrevemos
$$X\sim f(\lambda)$$
quando $X$ torna-se $f(\lambda)X$ sob uma transformação de escala. 

Dois tipos de lei de escala são especiais o suficiente para receberem nomes. Uma
grandeza é dita \emph{extensiva} quando sua lei de escala é proporcional, $X\sim
\lambda$, normalmente denotada por letras maiúsculas. Uma grandeza é dita \emph{
intensiva} quando sua lei de escala é constante, $y\sim1$, normalmente denotada
por letras minúsculas.

Note que a noção de escala é ambígua: poderiamos reformular todas as leis de
escala para usar $\lambda^2$, $1/\lambda$, $2^\lambda$ ou qualquer outra função
monotônica como parâmetro, e como consequência a noção de extensividade como
aqui apresentada está sujeita a essa mesma ambiguidade. Nas situações em que
essa confusão pode se manifestar, costuma-se utilizar diretamente alguma
grandeza de referência $\Lambda$ ao invés do parâmetro de escala no enunciado
dessas leis, onde fica implícita a extensividade de $\Lambda$, e portanto a de
qualquer grandeza para qual dizemos que $X\sim\Lambda$. De forma geral,
$\Lambda$ assim como as equações que regem as relações entre grandezas vão ser
tratadas como propriedades fixas do sistema em questão, enquanto as outras
grandezas serão tomadas como sujeitas a mudança, chamadas \emph{variáveis de
estado}.

\subsection{Calor}

A medida que um sistema torna-se complexo vamos perdendo a capacidade de
descrever e controlar precisamente as interações entre seus constituintes. Cada
um destes move-se de acordo com a dinâmica microscópica exercendo e estando
sujeito a forças dos outros de forma que os detalhes vão muito além da
especificidade de qualquer preparação experimental. Este movimento errático
dentro de um sistema físico é justamente o que nos permite caracterizar suas
propriedades através de leis de escala, uma vez que é responsavel por
redistribuir bolsões de certas grandezas ao longo de um sistema todo.

Quando lidamos especificamente com a energia, o efeito resultante destas várias
pequenas transferências de trabalho irregulares, o \emph{calor}, ainda que
sujeito à lei da conservação de energia, comporta-se de forma distinta do
trabalho que estamos acostumados na mecânica, e deve ser caracterizado como tal.
O calor é energia térmica em trânsito. Quando dois sistemas estão em uma
configuração que possibilita a troca de calor entre eles, são ditos em \emph{
contato térmico}. Um meio que permite a transferência de calor entre sistemas é
dito \emph{diatérmico}. Do contrário, um que não a permite, é dito \emph{
isolante}.

Da mesma maneira que o movimento irregular dos constituintes gera uma
redistribuição de grandezas dentro de um próprio sistema, ela pode também
acarretar algo semelhante entre sistemas, afinal a fronteira que separa um
sistema do outro quando estes são interagentes é uma construção teórica. Quando
dois sistemas encontram-se em uma situação onde a troca de calor entre eles é
possível, porém não ocorre de forma apreciável, são ditos em \emph{equilíbrio
térmico}. É fácil aceitar que um sistema esteja em equilíbrio térmico com uma
cópia de si mesmo, e até mesmo qualquer versão dele sujeita a uma transformação
de escala. A definição aqui dada também é inerentemente simétrica. Há no entanto
algo fundamental que precisa ser dito sobre a condição de equilíbrio.

\begin{law}
    Dados três sistemas $A$, $B$ e $C$. Se $A$ está em equilíbrio térmico com 
    $B$, e $B$ está em equilíbrio térmico com $C$, então $A$ está em equilíbrio
    térmico com $C$.
\end{law}

Esta lei da à noção de equilíbrio térmico uma estrutura de equivalência. Em
particular, nos permite definir a \emph{isoterma} de qualquer estado como a
colecão de todos os estados de todos os sistemas que estão em equilíbrio térmico
com o primeiro. Por sua vez, uma variável de estado $\theta$ que determina
unicamente a qual isoterma o estado pertence é chamada de \emph{escala de
temperatura}. 

Em breve veremos que existe uma noção de temperatura unificada que nos permite
identificar exatamente a qual isoterma um sistema pertence de uma forma global,
mas por enquanto uma noção localizada para cada sistema é mais do que
suficiente, e sua existência - assim como sua compatibilidade entre sistemas -
será tomada como pressuposto. 

\subsection{Trabalho e Energia}

Embora a transferência de calor seja uma característica emergente de sistemas
macroscópicos, ainda é possível realizar as transferências de energia que
estamos acostumados na mecânica usual, o \emph{trabalho}. Processos
termodinâmicos são capazes de levantar pesos, acelerar e frear trens, e esticar
a superfície de borracha de um balão. Estendemos a lei da conservação de energia
mecânica para o calor através da seguinte

\begin{law}
    Todo sistema termodinâmico possui uma variável de estado $U$, chamada de
    \emph{energia interna} tal que se num processo termodinâmico o sistema
    absorve calor $Q$ e recebe trabalho $W$, vale que 
    $$\Delta U=Q+W.$$
\end{law}

É fácil ver que a energia interna é deve ser uma variável extensiva. Para
processos contínuos, podemos dividílo em passos infinitesimais, e em cada um
deles escrevemos a primeira lei da termodinâmica como
$$dU=\delta Q+\delta W,$$
onde o uso do $\delta$ no lugar do $d$ indica que essas grandezas, apesar de
infinitesimais, são específicas
do processo ao qual o sistema está sendo submetido e não propriedades do sistema
em si. Isso torna-se material quando expandimos $\delta W$ em termos de \emph{
coordenadas de trabalho} ou \emph{volumes} $X_1,\dots,X_n$ que são variáveis
extensivas do sistema tais que 
$$\delta W=y_1dX_1+\cdots+y_ndX_n,$$
e $y_1,\dots,y_n$ são variáveis intensivas chamadas de \emph{coeficientes de
trabalho} ou \emph{pressões}. Nesta expansão vale ressaltar uma analogia com a
mecânica, onde cada coordenada corresponde a uma coordenada no espaço na qual um
corpo pode se mover, e cada coeficiente é a componente da força resultante na
direção correspondente. Os pares de variáveis $(X_1, y_1),\dots,(X_n,y_n)$ são
chamados pares de \emph{variáveis conjugadas}.

\subsection{Capacidade Térmica}

Se um sistema físico possui uma escala de temperatura $\theta$, é esperado que
este valor esteja sujeito a alterações a medida que o sistema sofre mudanças em
seu estado. Uma grandeza importante a ser definida para um processo
termodinâmico contínuo é sua \emph{capacidade térmica}, dado pela razão entre o
calor absorvido pelo sistema e a variação de sua temperatura:
$$C=\frac{\delta Q}{d\theta}.$$

Um processo específico de interesse geral é o processo \emph{isocórico}, onde
todas as coordenadas de trabalho são mantidas fixas, e a capacidade térmica
associada $C_V$ é uma variável de estado extensiva importante uma vez que dados
valores fixos para essas coordenadas, nos permite escrever a energia interna
como
$$U=\int C_V\,d\theta.$$
Além disso, vamos considerar que as escalas de temperatura são todas tais que
$C_V>0$. Outro processo onde é fácil entender a capacidade térmica como variável
de estado é o processo \emph{isobárico}, onde evoluimos o sistema mantendo todos
os coefficientes de trabalho constantes e permitimos as coordenadas variarem
para manter este vínculo. A capacidade térmica associada é denotada por $C_p$, e
vale que
$$H=U-(y_1X_1+\cdots+y_nX_n)=\int C_p\,d\theta.$$

\subsection{Problemas}

\begin{enumerate}
    \item Classifique as seguintes grandezas entre extensivas ou intensivas:
    \begin{enumerate}
        \item o número de moléculas de água em um copo
        \answer{extensiva}
        \item o índice de refração de determinado vidro
        \answer{intensiva}
        \item a densidade de determinado metal
        \answer{intensiva}
        \item o volume de gasolina no tanque de um carro
        \answer{extensiva}
    \end{enumerate}

    \item 
        A escala de temperatura Celsius é definida com $0\,\mathrm{^\circ C}$
        no ponto de fusão da água, $100\,\mathrm{^\circ C}$ no ponto de ebulição
        da água, e determinada pela dilatação de um filamento de álcool nos
        pontos intermediários. A escala Farenheit também é definida pela
        dilatação de um filamento de álcool, porém nela temos o ponto de fusão
        da água $32 \,\mathrm{^\circ F}$ e o de ebulição
        $212\,\mathrm{^\circ F}$. A qual temperatura em $\mathrm{^\circ C}$
        equivale a temperatura de $80\mathrm{^\circ F}$?
        \answer{
            $\theta_C=\frac{5\,\mathrm{^\circ C}}{9\,\mathrm{^\circ F}}
            (\theta_F-32\,\mathrm{^\circ F})= 27\,\mathrm{^\circ C}$
        }

    \item 
        A capacidade térmica molar do ouro é dada por $25{,}4\,\mathrm{J/mol
        \,^\circ C}$, a massa molar do ouro é $197\,\mathrm{g/mol}$ e sua
        densidade é $19{,}3\,\mathrm{g/cm^3}$. Dado que a Terra tem um raio de
        $6370\,\mathrm {km}$, qual a capacidade térmica de uma bola de ouro do
        tamanho da Terra?
        \answer{
            $C=\frac{4\pi\rho cR^3}{3\mu}=2{,}69\times10^{27}\,\mathrm{J/K}$
        }

    \item 
        A água tem um calor específico de $1\,\mathrm{cal/g\,^\circ C}=4{,}2
        \,\mathrm{J/g\,^\circ C}$. Quanto calor é necessário para aumentar a
        temperatura de $1\,\mathrm{kg}$ de água de $0\,\mathrm{^\circ C}$ até
        $100\,\mathrm{^\circ C}$?
        \answer{$Q=mc\Delta\theta=4{,}2\,\mathrm J$}

    \item 
        Considere dois objetos de capacidade térmica $C_1$ e $C_2$, com
        respectivas temperaturas inicias $\theta_1$ e $\theta_2$. Colocamos os
        dois em contato térmico e eventualmente obtemos equilíbrio. Qual a
        temperatura final dos dois? Comente sobre o caso $C_2\ll C_1$.
        \answer{
            $\theta=\frac{C_1\theta_1+C_2\theta_2}{C_1+C_2}\approx
            \theta_1+\frac{C_2}{C_1}(\theta_2-\theta_1)$
        }

    \item
        A capacidade térmica de certo objeto está no gráfico abaixo:
        \begin{figure}[H]
            \centering
            \begin{tikzpicture}[scale=2]
                \draw[step=0.2,black,thin]
                (0,0) grid (3,2);
                \draw[step=1,black,thick]
                    (0,0) grid (3,2);
                \draw
                    (1.5,-0.5) node {$\theta\,(\mathrm{^\circ C})$};
                \path
                    (0,0) node[below] {0} 
                    (1,0) node[below] {10} 
                    (2,0) node[below] {20} 
                    (3,0) node[below] {30};
                \draw
                    (-0.5,1) node[rotate=90] {$C\,(\mathrm{J/mol\,^\circ C})$};
                \path
                    (0,0) node[left] {0} 
                    (0,1) node[left] {10} 
                    (0,2) node[left] {20};
                \draw[ultra thick]
                    (0,0.8) -- (2.4,2);
            \end{tikzpicture}
        \end{figure}


        Qual o calor necessário para aumentar a temperatura do objeto de $10
        \,\mathrm{^\circ C}$ até $20\,\mathrm{^\circ C}$?
        \answer{$Q=\int_{\theta_i}^{\theta_f}C(\theta)\,d\theta=155\,\mathrm J$}

    \item
        (SOIF 2023) Considere um sistema físico de massa m cuja energia interna
        é dada pela expressão
        $$U=\frac{U_0}{e^{T_C/T}-1},$$
        em que $T_C$ representa uma temperatura característica do sistema, $U_0$
        é uma constante, e  $T$ a sua temperatura.
        \begin{enumerate}
            \item 
                Calcule a expressão do calor específico do material $c$ 
                \answer{
                    $c=\frac{dU}{dT}=
                    \frac{U_0T_Ce^{T_C/T}}{mT^2(e^{T_C/T}-1)^2}$
                }
            \item 
                Determine os valores limites quanto $T\ll T_C$ e $T\gg T_C$
                \answer{
                    $c\rightarrow\frac{U_0T_C}{mT^2}e^{-T_C/T},
                    \,c\rightarrow\frac{U_0}{mT_C}$
                }
            \item 
                Calcule o calor necessário para levar o sistema de $T_C$ até
                $2T_C$.
                \answer{$Q=\Delta U=\frac{U_0\sqrt e}{e-1}$}
        \end{enumerate}

    \item 
        (IPhO 1996) Uma peça de metal termicamente isolada é aquecida sob
        pressão atmosférica por uma corrente elétrica de forma que recebe
        energia elétrica a uma potência constante $P$. Isso leva ao aumento da
        temperatura absoluta $T$ do metal com o tempo $t$ como:
        $$T(t)=T_0\left(1+a\left(t-t_0\right)\right)^{1/4},$$
        onde $a$, $t_0$ e $T_0$ são constantes. Determine a capacidade térmica
        $C_p(T)$ do metal.
        \answer{$C=\frac{PT^3}{4aT_0^3}$}
\end{enumerate}

\section{O Gás Ideal}

\epigraph{\justifying [...] eu era o mais disposto a contestar algumas Objeções
de \emph{Francisco Lineu}, uma vez que o medo de admitir o \emph{Vácuo} tem
prevalecido em tantas pessoas proeminentes criadas pela aclamada Filosofia das
Escolas, que mesmo que discordem com ele e entre si sobre a solução para o
\emph{Fenômeno} do Experimento de \emph{Torricelli}, concordam em atribui-la a
uma substância extremamente rarefeita que preenche o espaço deixado pelo
Mercúrio.}{\emph{Robert Boyle}, A Defence of the Doctrine Touching the Spring
and Weight of the Air (1662)}

\noindent Entre meados do século XVII houveram esforços dirigidos a entender a
natureza do ar, que até o final do século XIX tornaram-se um estudo da natureza
dos gases em geral de um ponto de vista termodinâmico. Com as limitações
experimentais e teóricas da época, foi desenvolvido o modelo mais simples
possível de um sistema termodinâmico com algum grau de liberdade além da própria
quantidade de calor absorvida, o \emph{gás ideal}. 

Dentro do escopo do que era possível experimentalmente, determinou-se que até
uma ótima precisão, o volume $V$, a \emph{pressão} $p$ - força por unidade de
área que uma amostra de gás exerce nas paredes de seu recipiente - e o número de
moléculas $N$ de uma amostra de gás caracterizam completamente sua isoterma, 
$$\frac{pV}{N}\equiv RT,$$
onde $T$ é uma escala de temperatura conhecida como \emph{temperatura
termodinâmica}, cujo valor numérico é determinado pela constante dos gases
ideais $R$, hoje fixada em $R=8{,}3145\,\mathrm{J/mol\,K}$, de forma que o ponto
triplo da água (ponto de coexistência entre seus três estados físicos) ocorre a
uma temperatura de $273{,}16\,\mathrm K$. A equação de estado acima normalmente
é escrita da forma
$$pV=NRT,$$
e recebe o nome de \emph{lei dos gases ideais}.

\subsection{Trabalho e Energia}

Uma descrição da termodinâmica de um sistema não se esgota com as equações de
estado. É necessário também um entendimento das maneiras que o sistema pode
realizar trabalho, e também identificar o comportamento da sua energia interna.
Para isso, considere uma pequena expansão do gás de um volume $V$ para $V+dV$.
Independente do formato do recipiente, podemos dividir sua superfície em
pequenas facetas $i$, cada uma com área $A_i$, deslocando-se perpendicularmente
por uma distância $dx_i$, contribuindo uma variação de volume $dV_i=A_idx_i$ de
forma que $\sum_idV_i=dV$.

\begin{figure}[H]
    \centering
    \begin{tikzpicture}
        \foreach \n in {1,...,3} {
            \path[name path global=v0\n,spath/save global=v0\n] 
                (\n,0) ..
                controls (\n+0.3,1.5) and (\n+0.3,2.5)
                .. (\n,4);
            \path[name path global=h0\n,spath/save global=h0\n] 
                (0,\n-0.1) ..
                controls (1.5,\n+0.3) and (2.5,\n+0.3)
                .. (4,\n-0.1);
            \path[name path global=v\n,spath/save global=v\n] 
                (\n+0.3,0.3) ..
                controls (\n+0.6,1.9) and (\n+0.5,2.7)
                .. (\n+0.3,4.3); 
            \path[name path global=h\n,spath/save global=h\n] 
                (0.3,\n+0.2) ..
                controls (1.7,\n+0.8) and (2.9,\n+0.7)
                .. (4.3,\n+0.2); 
        }
        \foreach \x/\y/\n in {3/3/0,3/2/1,2/2/2,2/3/3} {
            \path[name intersections={of={h0\x} and {v0\y},by={p\n}}];
            \path[name intersections={of={h\x} and {v\y},by={q\n}}];
        }
        \fill[lightgray] 
            \foreach \p/\q/\r in {
                h03/v0/reverse,
                v02/h0/reverse,
                h02/v0/,
                v03/h0/
            } {[
                spath/split at intersections with={\p}{\q2},
                spath/split at intersections with={\p}{\q3},
                spath/get components of={\p}\c,
                spath/use={\getComponentOf{\c}2,weld,\r},
            ]};
        \foreach \n in {1,...,3} {
            \draw[thick,darkgray,spath/use=v0\n];
            \draw[thick,darkgray,spath/use=h0\n];
            \draw[dashed,spath/use=v\n];
            \draw[dashed,spath/use=h\n];
        }
        \foreach \n in {1,...,3} 
            \draw[->,thick] 
                (p\n) -- (q\n);
        \draw[->,thick] 
            (p0) node[below left=-2] {$A_i$} 
            -- (q0) node[above right=-2] {$dx_i$};
    \end{tikzpicture}
\end{figure}

Supondo a pressão aproximadamente constante durante essa expansão, a força
atuando em cada uma dessas facetas é, por definição $F_i=pA_i$, e portanto sua
expansão realiza trabalho $\delta W_i=pA_idx_i=pdV_i$. Assim, ao longo da
superfície toda, o gás realiza um trabalho $\sum_i\delta W_i=pdV$. Concluimos
portanto que a expressão para o trabalho infinitesimal em termos de variáveis de
estado para o nosso gás deve ser
$$\delta W_\text{expansão}=-pdV,$$
ou seja, $V$ é a coordenada de trabalho conjugada ao coeficiente de trabalho
$-p$. Isso nos permite escrever que o trabalho realizado por um gás em um
processo traçado em um diagrama de pressão por volume é dado pela área sob a
curva em questão.

Por sua vez, a energia interna do gás ideal é tomada como proporcional a
quantidade e temperatura, dada por
$$U=\frac{NRT}{\gamma-1}.$$
O parâmetro $\gamma$ recebe o nome de \emph{coeficiente de Poisson}, e para
gases nobres como hélio e neônio temos $\gamma=5/3$, enquanto para gases
diatômicos usuais como oxigênio e nitrogênio temos $\gamma=7/5$. 

\subsection{Processos Importantes}

Aqui vamos encontrar expressões para o trabalho realizado pelo gás e a
capacidade térmica de alguns processos de maior incidência e importância
teórica, assim como traçá-los em um diagrama de pressão por volume.

\subsubsection{Isocórico}

O processo isocórico é o processo contínuo em que mantemos as coordenadas de
trabalho (volume) constantes.
\begin{figure}[H]
    \centering
    \begin{tikzpicture}
        \draw [->, thick]
            (0,0) -- (5,0) node[below] {$V$};
        \draw [->, thick]
            (0,0) -- (0,4) node[left] {$p$};
        \draw
            (2.5,0.2) -- (2.5,-0.2) node[below] {$V_0$};
        \draw
            (0.2,1.2) -- (-0.2,1.2) node[left] {$p_i$};
        \draw
            (0.2,2.8) -- (-0.2,2.8) node[left] {$p_f$};
        \draw[ultra thick,postaction={mid arrow}]
            (2.5,1.2) -- (2.5,2.8);
        \fill
            (2.5,1.2) circle (0.05);
        \fill
            (2.5,2.8) circle (0.05);
    \end{tikzpicture}
\end{figure}
A cada passo vale que $dV=0$, e portanto o trabalho deste processo é $$W_\text{
isocórico}=0.$$ A capacidade térmica, por sua vez é
$$C_V=\frac{\delta Q}{dT}=\frac{dU+pdV}{dT}=\frac{NR}{\gamma-1}.$$

\subsubsection{Isobárico}

O processo isobárico é o processo contínuo em que mantemos os coeficientes de
trabalho (pressão) constantes.
\begin{figure}[H]
    \centering
    \begin{tikzpicture}
        \draw[->, thick] 
            (0,0) -- (5,0) node[below] {$V$};
        \draw[->, thick]
            (0,0) -- (0,4) node[left] {$p$};
        \draw 
            (1.5,0.2) -- (1.5,-0.2) node[below] {$V_i$};
        \draw
            (3.5,0.2) -- (3.5,-0.2) node[below] {$V_f$};
        \draw
            (0.2,2) -- (-0.2,2) node[left] {$p_0$};
        \draw[ultra thick,postaction={mid arrow}]
            (1.5,2) -- (3.5,2);
        \fill
            (1.5,2) circle (0.05);
        \fill
            (3.5,2) circle (0.05);
    \end{tikzpicture}
\end{figure}
A pressão no processo todo é constante, e portanto a integral do trabalho é
simplesmente
$$W_\text{isobárico}=\int_{V_i}^{V_f}p\,dV=p_0(V_f-V_i).$$
A capacidade térmica torna-se
$$C_p=\frac{\delta Q}{dT}=\frac{dU+pdV}{dT}=\frac{\gamma NR}{\gamma-1}
=\gamma C_V = C_V+NR.$$

\subsubsection{Isotérmico}

O processo isotérmico é o processo contínuo em que mantemos a temperatura
constante. Para o gás ideal isso significa simplesmente que $pV=\text{const.}$
\begin{figure}[H]
    \centering
    \begin{tikzpicture}
        \draw[->, thick]
            (0,0) -- (5,0) node[below] {$V$};
        \draw[->, thick]
            (0,0) -- (0,4) node[left] {$p$};
        \draw
            (1.5,0.2) -- (1.5,-0.2) node[below] {$V_i$};
        \draw
            (3.5,0.2) -- (3.5,-0.2) node[below] {$V_f$};
        \draw
            (0.2,1.2) -- (-0.2,1.2) node[left] {$p_f$};
        \draw
            (0.2,2.8) -- (-0.2,2.8) node[left] {$p_i$};
        \draw[ultra thick,postaction={mid arrow},samples=10,domain=1.5:3.5]
            plot (\x,{4.2/\x});
        \fill
            (1.5,2.8) circle (0.05);
        \fill
            (3.5,1.2) circle (0.05);
    \end{tikzpicture}
\end{figure}
A expressão para o trabalho fica mais simétrica quando expressada em termos da
temperatura de operação $T_0$:
$$W_\text{isotérmico}=\int_{V_i}^{V_f}p\,dV=NRT_0\int_{V_i}^{V_f}\frac{1}{V}
\,dV= NRT_0\log\frac{V_f}{V_i}=NRT_0\log\frac{p_i}{p_f}.$$
Como não há variação de temperatura, a capacidade térmica diverge.

\subsubsection{Isotérmico}

O processo adiabático é o processo contínuo em que mantemos o sistema isolado,
sem perimitir trocas de calor.
\begin{figure}[H]
    \centering
    \begin{tikzpicture}
        \draw[->, thick]
            (0,0) -- (5,0) node[below] {$V$};
        \draw[->, thick]
            (0,0) -- (0,4) node[left] {$p$};
        \draw
            (1.5,0.2) -- (1.5,-0.2) node[below] {$V_i$};
        \draw
            (3.5,0.2) -- (3.5,-0.2) node[below] {$V_f$};
        \draw
            (0.2,0.83) -- (-0.2,0.83) node[left] {$p_f$};
        \draw
            (0.2,3.4) -- (-0.2,3.4) node[left] {$p_i$};
        \draw[ultra thick,postaction={mid arrow},samples=10,domain=1.5:3.5]
            plot (\x,{3.4*(1.5/\x)^(5/3)});
        \fill
            (1.5,3.4) circle (0.05);
        \fill
            (3.5,0.83) circle (0.05);
    \end{tikzpicture}
\end{figure}
Partindo da primeira lei $dU+pdV=0$ e da equação de estado $pV=NRT$ podemos
deduzir três relações que se realizam no processo adiabático:
\begin{align*}
    \frac{dp}{p}+\gamma\frac{dV}{V}=0&\Leftrightarrow pV^\gamma=\text{const.}\\
    \frac{dT}{T}+(\gamma-1)\frac{dV}{V}=0&\Leftrightarrow TV^{\gamma-1}=
    \text{const.}\\
    \gamma\frac{dT}{T}-(\gamma-1)\frac{dp}{p}=
    0&\Leftrightarrow T^\gamma/p^{\gamma-1}=\text{const.}\\
\end{align*}
Note que no diagrama acima com $\gamma=5/3$ a curva é mais inclinada do que a
isoterma do diagrama anterior. Como não há troca de calor, podemos calcular o
trabalho realizado diretamente pela variação de energia interna,
$$W_\text{adiabático}=U_i-U_f=\frac{NR(T_f-T_i)}{\gamma-1}
=\frac{p_iV_i-p_fV_f}{\gamma-1}.$$
Como não há calor absorvido, a capacidade térmica é zero.

\subsection{Problemas}

\begin{enumerate}
    \item
        A lei de Stevin relaciona a variação de pressão $p$ de um fluido de
        densidade $\rho$ sob ação da gravidade $g$ com a variação de altura
        $h$ como
        $$dp=-\rho gdh.$$
        \begin{enumerate}
            \item 
                Considere o fluido como sendo um gás ideal de massa molar
                $\mu$. Encontre a densidade $\rho$ como função da pressão $p$ e
                da temperatura $T$.
                \answer{$\rho=\frac{\mu p}{RT}$}
            \item
                Partindo da Lei de Stevin, escreva uma equação para como a
                pressão varia com a altura.
                \answer{$\frac{dp}{dh}=-\frac{\mu g}{RT}p$}
            \item 
                Seja $p_0$ a pressão atmosférica no nível do mar, marcado como
                $h=0$. Supondo que a temperatura da atmosfera seja constante com
                a altura, encontre $p(h)$.
                \answer{$p=p_0e^{-\mu gh/RT}$}
            \item 
                O monte Everest possui uma altura de $8848\,\mathrm{m}$ com
                relação ao nível do mar. Suponha que a temperatura do ar é
                constante $T=273\,\mathrm{K}$ e a gravidade $g=9{,}79
                \,\mathrm{m/s^2}$. Dado que a pressão no topo do everest é
                $0{,}333p_0$, estime a massa molar $\mu$ do ar atmosférico.
                \answer{
                    $\mu=\frac{RT}{gh}\log\left(\frac{p_0}{p}\right)=28.8
                    \,\mathrm{g/mol}$
                }
            \item 
                Por fim, supondo que o ar seja composto de gás nitrogênio 
                $\mathrm{N_2}$ e oxigênio $\mathrm{O_2}$, encontre a proporção
                $x$ de oxigênio na atmosfera.
                \answer{
                    $x=\frac{\mu-\mu_\mathrm{N_2}}
                    {\mu_\mathrm{O_2}-\mu_\mathrm{N_2}}=20\%$
                }
        \end{enumerate}

    \item
        (SOIF 2001) Um cilíndro cheio de gás tem a área da seção reta igual a
        $A=10\,\mathrm{cm^2}$ e é fechado por um êmbolo pesado e móvel. O
        cilíndro começa a subir com aceleração de $a=20\,\mathrm{m/s^2}$ e,
        mantendo a temperatura do gás constante, seu volume fica $\eta=2/3$ do
        original. Ache a massa $M$ do êmbolo sabendo que a gravidade local é $g=
        10\,\mathrm{m/s^2}$ e a pressão atmosférica é $p_0=10^5
        \,\mathrm{N/m^2}$. 
        \answer{$M=\frac{(1-\eta)p_0A}{\eta a-(1-\eta)g}=3{,}3\,\mathrm{kg}$}

    \item
        (Irodov) Defina a taxa de evacuação $r$ de um gás como o volume perdido
        por unidade de tempo quando medido à própria pressão e temperatura do
        gás. Encontre a pressão $p$ como função do tempo $t$ para um gás dentro
        de um recipiente de volume $V$ que expulsa gás a uma taxa de evacuação
        constante $r$. Suponha que a temperatura do gás não muda durante o
        processo e que a pressão inicial é $p_0$.
        \answer{$p=p_0e^{-rt/V}$}

    \item
        Considere um cilíndro vertical fechado em baixo e tampado por um pistão
        móvel de massa $M$. Dentro do cilíndro há uma massa $m$ de gás ideal de
        massa molar $\mu$. Sendo $g$ a aceleração da gravidade e supondo que
        fora do cilíndro há vácuo, encontre a altura $H$ do pistão móvel como
        função da temperatura $T$ do gás.
        \answer{$H=\frac{RT}{\mu g}\log\left(1+\frac{m}{M}\right)$}

    \item
        (SOIF 2005) Um cilíndro com paredes termicamente isoladas contém duas
        espécies de gás distintas, separadas por uma parede isolante. O primeiro
        gás tem coeficiente de Poisson $\gamma_1$, pressão $p_1$, volume $V_1$ e
        temperatura $T_1$. O segundo tem respectivos $\gamma_2,p_2,V_2$ e $T_2$.
        Como fica a pressão $p$ e temperatura $T$ no cilíndro após retirarmos a
        parede interna e esperarmos o equilíbrio termodinâmico?
        \answer{
            $T=\frac{\frac{p_1V_1}{\gamma_1-1}+\frac{p_2V_2}{\gamma_2-1}}
            {\frac{p_1V_1}{T_1(\gamma_1-1)}+\frac{p_2V_2}{T_2(\gamma_2-1)}},
            \quad p=\frac{\frac{p_1V_1}{T_1}+\frac{p_2V_2}{T_2}}{V_1+V_2}T$
        }

    \item
        Um tubo vertical liso possui dois raios, com a área superior maior do
        que a inferior por uma parcela $A$, equipado com um par de pistões de
        massa total $M$, como na figura abaixo.
        \begin{figure}[H]
            \centering
            \begin{tikzpicture}[scale=0.7]
                \fill[lightgray]
                    (1.5,1.5) -- (-1.5,1.5) -- (-1.5,0) -- (-1,0)
                    -- (-1,-1.5) -- (1,-1.5) -- (1,0) -- (1.5,0) -- (1.5,1.5);
                \fill[darkgray]
                    (-1.5,2) -- (1.5,2) -- (1.5,1.5) -- (-1.5,1.5);
                \draw[thick]
                    (-1.5,2) -- node[pos=0.5,above] {$p_0$} (1.5,2)
                    -- (1.5,1.5) -- (-1.5,1.5) -- (-1.5,2);
                \fill[darkgray]
                    (-1,-2) -- (1,-2) -- (1,-1.5) -- (-1,-1.5);
                \draw[thick]
                    (-1,-2) -- node[pos=0.5,below] {$p_0$} (1,-2)
                    -- (1,-1.5) -- (-1,-1.5) -- (-1,-2);
                \fill
                    (-1.5,3) -- (-1.5,0) -- (-1,0) -- (-1,-3) -- (-1.2,-3)
                    -- (-1.2,-0.2) -- (-1.7,-0.2) -- (-1.7,3);
                \draw[thick]
                    (-1.5,3) -- (-1.5,0) -- (-1,0) -- (-1,-3);
                \draw[thick]
                    (1.5,3) -- (1.5,0) -- (1,0) -- (1,-3);
                \fill
                    (1.5,3) -- (1.5,0) -- (1,0) -- (1,-3) -- (1.2,-3)
                    -- (1.2,-0.2) -- (1.7,-0.2) -- (1.7,3);
                \draw[ultra thick]
                    (0,1.5) -- (0,-1.5);
                \draw[ultra thick,->]
                    (-2.3,1) -- node[pos=0.5,left] {$g$} (-2.3,-1);
            \end{tikzpicture}
        \end{figure}
        Cada pistão move-se na seção correspondente, e o menor possui massa $m$
        e área $a$. Uma quantidade $N$ de gás ideal de massa desprezível
        encontra-se entre os pistões, e estes estão presos um ao outro por um
        fio inextensível de comprimento $L$. A pressão externa aos tubos é $p_0$
        e a gravidade local é $g$. Inicialmente, a temperatura do gás é baixa,
        de forma que o pistão superior está apoiado na junção das duas seções, e
        o fio encontra-se frouxo.
        \begin{enumerate}
            \item
                Encontre a temperatura $T_0$ a partir da qual o fio fica
                tensionado
                \answer{$T_0=\frac{(p_0a-mg)L}{NR}$}
            \item
                Encontre a temperatura $T_1$ a partir da qual o pistão superior
                começa a subir. Esta é maior ou menor do que $T_0$?
                \answer{$T_1=\frac {(p_0A+Mg)L}{NR}>T_0$}
            \item
                Encontre a temperatura máxima $T_\text{máx}$ a partir da qual o
                pistão de baixo descarrilha de sua seção e o gás vaza para o
                meio externo
                \answer{$T_\text{máx}=T_1\left(1+\frac{a}{A}\right)$}
            \item
                Faça um gráfico da pressão $p$ como função da temperatura para
                este sistema
                \answer{
                    \begin{tikzpicture}
                        \draw[very thick,->]
                            (0,0) -- (3,0) node[below] {$T$};
                        \draw[very thick,->]
                            (0,0) -- (0,2) node[above] {$p$};
                        \draw[ultra thick]
                            (0,0.6) -- (0.4,0.6) -- (1.3,1.5) -- (2.3,1.5);
                        \filldraw[fill=white]
                            (2.3,1.5) circle (0.05);
                        \draw
                            (0.1,0.6) -- (-0.1,0.6) node[left] {$p_0-mg/a$};
                        \draw
                            (0.1,1.5) -- (-0.1,1.5) node[left] {$p_0+Mg/A$};
                        \draw
                            (0.4,0.1) -- (0.4,-0.1) node[below] {$T_0$};
                        \draw
                            (1.3,0.1) -- (1.3,-0.1) node[below] {$T_1$};
                        \draw
                            (2.3,0.1) -- (2.3,-0.1) node[below] 
                            {$T_\text{máx}$};
                    \end{tikzpicture}
                }
        \end{enumerate}

    \item
        Mostre que a para um processo num gás ideal
        $$pV^\alpha=\text{const}\Leftrightarrow C=\text{const}.$$
        encontrando uma relação entre $C$ e $\alpha$ para $N$ mols de um gás
        ideal de coeficiente de Poisson $\gamma$
        \answer{$C=NR\left(\frac{1}{\gamma-1}-\frac{1}{\alpha-1}\right)$}

    \item
        (Irodov) Certa quantidade de gás ideal de coeficiente de Poisson
        $\gamma$ realiza um processo em que sua capacidade térmica é função da
        temperatura dada por $C=p_0V_0/T$. Encontre uma relação entre a pressão
        e o volume do gás nesse processo, sabendo que ele passa pelo ponto
        $(p_0,V_0,T_0)$.
        \answer{$pV^\gamma=p_0V_0^\gamma e^{\left(1-\frac{p_0V_0}{pV}\right)}$}

    \item
        Um recipiente isolado termicamente é dividido por um pistão que pode
        movimentar-se sem atrito, conforme a figura abaixo.
        \begin{figure}[H]
            \centering
            \begin{tikzpicture}
                \fill 
                    (0,0) -- (5,0) -- (5,2) -- (0,2) -- (0,0)
                    -- (-0.2,-0.2) -- (-0.2,2.2) -- (5.2,2.2)
                    -- (5.2,-0.2) -- (-0.2,-0.2) -- (0,0);
                \fill[lightgray]
                    (0,0) -- (2,0) -- (2,2) -- (0,2) -- (0,0);
                \fill[darkgray]
                    (2,0) -- (2,2) -- (2.2,2) -- (2.2,0) -- (2,0);
                \draw[very thick]
                    (2,0) -- (2,2) -- (2.2,2) -- (2.2,0) -- (2,0);
                \draw[decoration={
                    coil,
                    segment length=5,
                    aspect=0.3,
                    amplitude=8,
                    pre length=10,
                    post length=10
                },decorate,ultra thick]
                    (2.2,1) -- (5,1);
            \end{tikzpicture}
        \end{figure}
        A parte da esquerda é preenchida com um mol de gás ideal monoatômico, e
        a parte da direita encontra-se evacuada. O pistão é conectado à parede
        da direita por meio de uma mola cujo comprimento livre é igual ao
        comprimento total do recipiente. Determine o calor especifico molar $c$
        do gás sob essas condições em função da constante dos gases ideais $R$.
        \answer{$c=2R$}

    \item
        Um gás ideal de coeficiente de Poission $\gamma$ com $N$ moléculas é
        sujeito aos seguintes processos:
        \begin{enumerate}
            \item
                $p=p_0(1-V/V_0)$
                \answer{
                    $T=\frac{T_0}{4},\,c=R\left(\frac{1}{\gamma-1}+
                    \frac{V_0-V}{V_0-2V}\right)$
                }
            \item
                $p=p_0(1-(V/V_0)^2)$
                \answer{
                    $T=\frac{2T_0}{3\sqrt3},\,c=R\left(\frac{1}{\gamma-1}+
                    \frac{V_0^2-V^2}{V_0^2-3V^2}\right)$
                }
            \item
                $p=p_0e^{-V/V_0}$
                \answer{
                    $T=\frac{T_0}{e},\,c=R\left(\frac{1}{\gamma-1}+
                    \frac{V_0}{V_0-V}\right)$
                }
        \end{enumerate}
        Onde $p_0$ e $V_0$ são constantes positivas. Encontre a temperatura
        máxima em termos de $T_0=\frac{p_0V_0}{NR}$ e o calor específico molar
        como função do volume $V$ e das constantes $\gamma$, $R$, e $V_0$.

    \item
        (SOIF 2004) Considere um cilíndro hermeticamente vedado com paredes
        adiabáticas, fechado em ambas as extremidades e dividido em duas partes
        por um êmbolo isolante que pode mover-se livremente sem atrito.
        Inicialmente o volume, pressão e temperatura de gases em cada parte do
        cilíndro é $V_0,p_0$ e $T_0$, respectivamente. No lado direito do
        cilíndro é colocado um aquecedor que aquece o gás até que sua pressão
        atinja $\alpha p_0$. Considerando os gases ideais de coeficiente de
        Poisson $\gamma$, encontre
        \begin{enumerate}
            \item
                O volume final do lado esquerdo
                \answer{$V_L=V_0\alpha^{-\frac{1}{\gamma}}$}
            \item
                a temperatura final do lado esquerdo
                \answer{$T_L=T_0\alpha^{1-\frac{1}{\gamma}}$}
            \item
                a temperatura final do lado direito
                \answer{
                    $T_R=T_0\alpha\left(2-\alpha^{-\frac{1}{\gamma}}\right)$
                }
            \item
                o trabalho efetuado sobre o gás do lado esquerdo
                \answer{$W=p_0V_0\left(\alpha^{1-\frac{1}{\gamma}}-1\right)$}
        \end{enumerate}

    \item
        Considere um longo cilíndro horizontal semiaberto. Coloque-o para girar,
        com seu eixo na vertical posicionado no lado aberto, a uma velocidade
        angular $\omega$. Encontre a pressão $p$ do ar dentro do tubo como
        função da distância $r$ do eixo de rotação. Suponha o ar um gás ideal de
        temperatura $T$ e massa molar $\mu$, e que a pressão atmosférica é $p_0$.
        \answer{$p=p_0e^\frac{\mu\omega^2r^2}{2RT}$}

    \item
        Um recipiente contém $N$ móls de hélio. Este gás passa por um processo
        termodinâmico em que sua capacidade térmica depende da temperatura como
        $C=\frac{3NRT}{4T_0}$, sendo $T_0$ a temperatura inicial do gás.
        Encontre o trabalho realizado sobre o gás ao longo deste processo até
        que ele seja comprimido ao seu volume mínimo.
        \answer{$W=\frac{3}{8}NRT_0$}

    \item
        (IPhO 1996) Considere duas bolas metálicas idênticas. Uma está pendurada
        no teto pela superfície enquanto a outra está apoiada no chão, como na
        figura abaixo.
        \begin{figure}[H]
            \centering
            \begin{tikzpicture}
                \draw[thick]
                    (-2,2) -- (-2,0.8);
                \filldraw[fill=gray]
                    (-2,0.2) circle (0.6);
                \draw[ultra thick]
                    (-3,2) -- (-1,2);
                \filldraw[fill=gray]
                    (2,-0.4) circle (0.6);
                \draw[ultra thick]
                    (3,-1) -- (1,-1);
            \end{tikzpicture}
        \end{figure}
        Uma mesma pequena quantidade de calor é fornecida para ambas as bolas.
        Qual delas fica com uma temperatura maior?
        \answer{a bolinha pendurada}

    \item
        Considere um recipiente com paredes isolantes e rígidas inicialmente
        evacuado. Faz-se um pequeno furo no recipiente e ar atmosférico entra
        lentamente. Encontre a temperatura do ar no interior do balão no momento
        em que o fluxo de ar para dentro cessa. O ar atmosférico pode ser
        considerado um gás ideal de coeficiente de Poisson $\gamma$ e
        temperatura $T_0$.
        \answer{$T=\gamma T_0$}

    \item
        O método de Rüchhardt para medir o coeficiente de Poisson do ar consiste
        no seguinte. Uma garrafa de volume $V_0$ é preenchida com ar
        atmosférico. Seu gargalo tem área $A$ e neste é colocado uma bola justa
        de massa $m$, solta a partir do repouso. Dado que a gravidade local é
        $g$ e a pressão atmosférica é $p_0$.
        \begin{figure}[H]
            \centering
            \begin{tikzpicture}[scale=1.8]
                \fill[lightgray]
                    (0,0.8) circle (0.2);
                \draw[thick]
                    (0,0.8) node {$m$} circle (0.2);
                \draw[thick]
                    (-0.2,1) -- (-0.2, 0) 
                    arc (105:435:{0.2/sin(15)}) -- (0.2,1);
                \draw
                    (0,{-0.2/sin(15)}) node {$V_0$};
                \draw[ultra thick,->]
                    (-1.1,0.5) -- node[pos=0.5,left] {$g$} (-1.1,-0.5);
            \end{tikzpicture}
        \end{figure}

        Desprezando atritos e trocas de calor entre a garrafa e o meio externo e
        supondo que o volume do gargalo é muito menor do que o do resto da
        garrafa, encontre
        \begin{enumerate}
            \item
                A distância máxima $L$ que a bola afunda no gargalo após ser
                solta
                \answer{$L=\frac{2mgV_0}{\gamma A^2p_0}$}
            \item
                O período $\tau$ do movimento subsequente
                \answer{$\tau=\frac{2\pi}{A}\sqrt\frac{mV_0}{\gamma p_0}$}
        \end{enumerate}
        Pode ser útil que
        $$|nx|\ll1\Rightarrow(1+x)^n\approx1+nx.$$

    \item
        (200 PPP) Considere um cilíndro isolado fechado que possui uma parede
        condutora móvel em seu interior que separa duas quantidades iguais de
        gás ideal de coeficiente de Poisson $\gamma$. Se aplicarmos uma força 
        externa constante muito grande nesta parede e esperarmos o equilíbrio
        térmico, como ficará a razão entre os volumes das partes expandidas e
        contraidas? 
        \answer{$r=\frac{\sqrt\gamma+1}{\sqrt\gamma-1}$}

    \item
        Um cilíndro vertical isolante possui a base fechada e dois pistões de
        peso desprezível. Entre a base e o primeiro pistão, diatérmico, existe
        certa quantidade de hélio. Entre o primeiro e o segundo pistão,
        isolante, existe certa quantidade de hidrogênio. Inicialmente, o volume
        de hidrogênio é 1/3 do volume de hélio. Dá se uma quantidade de calor
        $Q$ para o hélio e o pistão superior move-se para cima uma distância
        $D$. Após certo tempo, há outra movimentação do pistão superior. Quanto
        ele se moveu e para qual direção?
        \answer{cai de $D/7$}

    \item
        Um fino tubo circular de raio $r$ é disposto como na figura abaixo.
        \begin{figure}[H]
            \centering
            \begin{tikzpicture}[scale=0.7]
                \fill[lightgray,even odd rule]
                    (0,0) circle (2.3) (0,0) circle (2.7);
                \draw[very thick]
                    (0,0) circle (2.3);
                \draw[very thick]
                    (0,0) circle (2.7);
                \draw[ultra thick]
                    (0,-2.7) -- (0,-2.3);
                \fill
                    ({2.5*sin(20)},{2.5*cos(20)}) circle (0.2);
                \draw[thick,dashed]
                    (0,2.3) -- node[pos=0.4,left] {$r$} 
                    (0,0) -- ({2.3*sin(20)},{2.3*cos(20)});
                \draw[ultra thick,->]
                    (-3.2,1) -- node[pos=0.5,left] {$g$} (-3.2,-1);
            \end{tikzpicture}
        \end{figure}
        Na parte inferior do tubo há uma parede, e dentro dele há uma bolinha de
        massa $m$ que pode se deslocar sem atrito ao longo do tubo, separando-o
        em duas partes ambas contendo uma quantidade $N$ de gás ideal. Quando a
        temperatura do sistema é alta, a bolinha tende a ficar equilibrada no
        ápice do tubo. No entanto quando a temperatura cai abaixo de certo valor
        $T_c$, este equilíbrio mecânico torna-se instável e a bolinha cai para
        algum dos lados e encontra um novo ponto de equilíbrio.
        \begin{enumerate}
            \item
                Encontre a temperatura crítica $T_c$
                \answer{$T_c=\frac{\pi^2mgr}{2NR}$}
            \item
                A capacidade térmica do sistema é descontínua em $T_c$. Encontre
                a diferença de capacidade térmica $\Delta C$ entre temperaturas
                um pouco menores que $T_c$ e um pouco maiores que $T_c$. 
                \answer{$\Delta C=\frac{NR}{1+\pi^2/6}$}
        \end{enumerate}
        Pode ser útil que
        $$|x|\ll1\Rightarrow\sin x\approx x-\frac{1}{6}x^3.$$
\end{enumerate}

\section{Entropia}

\epigraph{\justifying Para encontrar um nome para esta função, Clausius disse
\begin{displayquote}
    Eu prefiro me basear em línguas antigas para o nome de grandezas científicas
    importantes, de forma que elas signifiquem a mesma coisa em todas as línguas
    vivas. Assim, eu proponho chamar $S$ de \emph{entropia} de um corpo, da
    palavra grega "transformação". Eu intencionalmente cunhei a palavra \emph{
    entropia} de forma a ser similar a energia, dado que essas duas grandezas
    são tão análogas em seu significado físico que uma analogia de denominação
    me parece muito útil.
\end{displayquote}
Com isso, ao contrário do que aconteceria se extraisse um nome do corpo da
linguagem contemporânea (como calor perdido), ele conseguiu cunhar um termo que
tem o mesmo significado para todo mundo: nenhum.}{\emph{Leon N. Cooper}, An
Introduction to the Meaning and Structure of Physics (1968)}

\subsection{Ciclos termodinâmicos}

Dado um sistema físico, podemos realizar uma sequência de processos
termodinâmicos encadeados que eventualmente resultam no sistema retornando ao
seu estado inicial, isto é, de forma que todas as variáveis de estado do sistema
sejam iguais antes e depois da sequência de processos. Tal sequência de
processos recebe o nome de \emph{ciclo termodinâmico}. Denotando cada um dos
processos por um índice $i$, podemos também denotar o calor $Q_i$ e trabalho
$W_i$ recebidos pelo sistema, que totalizam
$$Q_\text{tot}=\sum_iQ_i,\quad W_\text{tot}=\sum_iW_i,$$
e da primeira lei da termodinâmica devemos concluir que
$$\Delta U=Q_\text{tot}+W_\text{tot}=0.$$

De maneira geral vamos sempre supor que a divisão em processos $i$ é refinada o
suficiente de forma que o sinal de $Q_i$ caracteriza o processo no seguinte
sentido. Se $Q_i>0$ o processo apenas absorve calor, e se $Q_i<0$ o processo
apenas libera calor. Assim, podemos separar o calor em entradas e saídas como
$$Q_\text{in}=\sum_{Q_i>0}Q_i,\quad Q_\text{out}=-\sum_{Q_i<0}Q_i$$
e escrever $Q_\text{tot}=-W_\text{tot}=Q_\text{in}-Q_\text{out}$. Por sua vez, o
sinal de $W_\text{tot}$ será utilizado para dividir ciclos termodinâmicos em
dois grandes grupos. Quando $W_\text{tot}<0$, estamos utilizando energia na
forma de calor para realizar trabalho num meio externo, sendo um \emph{ciclo de 
potência} ou \emph{máquina térmica}. Quando $W_\text{tot}>0$ estamos
utilizando energia na forma de trabalho para trasferir calor num meio externo,
sendo um \emph{ciclo de refrigeração} ou \emph{bomba de calor}. Ambas as classes
de ciclos possuem noções de eficiência correspondentes. Para máquinas térmicas 
definimos o \emph{rendimento}
$$\eta=-\frac{W_\text{tot}}{Q_\text{in}},$$
enquanto para bombas de calor definimos o \emph{coeficiente de performance}, e a
noção vai depender se o ciclo é usado como refrigerador, onde o calor de
interesse é o absorvido
$$\mu_\text{cool}=\frac{Q_\text{in}}{W_\text{tot}},$$
ou como aquecedor, onde o calor de interesse é o liberado
$$\mu_\text{heat}=\frac{Q_\text{out}}{W_\text{tot}}=\mu_\text{cool}+1.$$

\subsection{A Segunda Lei}

Aqui começamos a lidar com a possibilidade de processos físicos que não podem
ser "desfeitos" de uma maneira trivial. Um mergulhador olímpico pode saltar e
cair na piscina, dissipando sua energia na água, mas não vemos a água se
organizando para empurrá-lo de volta para a prancha. Um balão de hélio ao
estourar espalha suas moléculas pela atmosfera, mas não vemos elas voltando para
a região original e expulsando o ar em volta de maneira espontânea. Uma cápsula
espacial caindo pela atmosfera é freada pela resistência do ar e entra em
chamas, mas não vemos o ar absorvendo essas chamas e catapultando a cápsula de
volta para a órbita. 

Todos os processos acima são chamados \emph{irreversíveis}, ou seja, apesar de
ser possível retornar o sistema de interesse para o estado em que estava antes
do processo, é necessária a intervenção de algum outro sistema, normalmente na
forma de trabalho. O mergulhador precisa subir novamente as escadas da prancha.
Precisamos bombear uma quantidade de hélio para dentro de um balão. Precisamos
de um foguete para lançar a cápsula espacial. 

Enquanto a todos os processos macroscópicos na natureza são irreversíveis, a
estrutura matemática da termodinâmica se beneficia muito ao considerarmos
processos idealizados \emph{reversíveis}. São aqueles que, se levam o sistema de
um estado $i$ até um estado $f$ absorvendo calor $Q$ e recebendo trabalho $W$,
existe um processo que leva do estado $f$ até o estado $i$ liberando calor $Q$
(absorvendo $-Q$) e realizando trabalho $W$ (recebendo $-W$). Note que um ciclo
termodinâmico será reversível quando todos os processos envolvidos são
reversíveis também.

Se definirmos um \emph{reservatório térmico} como um sistema termodinâmico tão
grande, a ponto de que sua capacidade térmica seja efetivamente infinita nas
escalas de energia que estamos interessados, podemos introduzir a seguinte lei
física enunciada por Kelvin em 1851, e reformulada por Planck em 1903:
\begin{law}[Kelvin-Planck]
    Não existe uma máquina térmica que consegue operar trocando calor com um
    único reservatório térmico.
\end{law}
Ou seja, se um sistema realiza algum ciclo termodinâmico em contato com um
único reservatório, o sistema necessariamente deve absorver trabalho ao invés de
realizá-lo.

O que acontece então no caso de dois reservatórios? Vamos considerar um ciclo
que em cada passo pode trocar calor com um de dois reservatórios a temperaturas
$\theta_C$ e $\theta_H$, com os nomes escolhidos de forma que o calor total
recebido de $\theta_H$ pelo sistema é positivo. Então vale o seguinte

\begin{theorem}[Carnot]
    Existe uma função única $g$ da temperatura do reservatório de forma que o
    rendimento de um ciclo termodinâmico satisfaz
    $$\eta\leq1-\frac{g(\theta_C)}{g(\theta_H)}.$$
    Temos igualdade no caso em que o ciclo é reversível, recebendo o nome de
    cíclo de Carnot.
\end{theorem}
Note que a função $g$ é uma escala de temperatura, que recebe o nome especial de
\emph{temperatura termodinâmica} $T=g(\theta)$. É fácil ver que dois
reservatórios com diferentes sinais para a temperatura levam a uma violação da
Segunda Lei, e portanto vamos convencionar que todos eles possuem temperatura
positiva. Note que o teorema de Carnot também nos mostra que a troca de calor
entre dois sistemas a temperaturas termodinâmicas diferentes é sempre
irreversível, e também a uma formulação equivalente da Segunda Lei como

\setcounter{law}{1}
\begin{law}[Clausius]
    Não existe um processo termodinâmico cujo único resultado é a transferência
    de calor de um corpo frio para um corpo quente.
\end{law}

O resultado generaliza imediatamente para muitos reservatórios, e é também
devido ao
\begin{theorem}[Clausius]
    Se um ciclo termodinâmico possui passos $i$ onde recebe calor $Q_i$ de um
    reservatório à temperatura $T_i$ vale que
    $$\sum_i\frac{Q_i}{T_i}\leq0,$$
    com igualdade num ciclo reversível.
\end{theorem}

Prestemos muita atenção no caso de igualdade. Ele nos mostra que para qualquer
sequência de processos reversíveis, onde precisamos que a temperatura do sistema
seja igual à do reservatório de onde recebe calor, o somatório do teorema zera.
Podemos enfim concluir que deve existir uma variável de estado $S$, chamada
\emph{entropia}, definida para qualquer sistema termodinâmico de forma que, em
qualquer isoterma $T$ onde o sistema absorve calor $Q$,
$$\Delta S=\frac{Q}{T}.$$
No caso de processos contínuos gerais, que são sempre reversíveis, a expressão
acima vale em forma diferencial
$$dS=\frac{\delta Q}{T}.$$
Podemos também escrever o enunciado moderno da Segunda Lei, que também é o mais
conhecido no meio leigo apesar da elusividade da definição de entropia em si.
\setcounter{law}{1}
\begin{law}[Entropia]
    Para um sistema isolado, os únicos processos termodinâmicos possíveis são
    aqueles que acarretam um aumento da entropia total. Um processo é reversível
    se, e somente se não acarreta em variação da entropia total.
\end{law}

Munidos com a definição de entropia, podemos também obter uma representação
gráfica para um cíclo reversível de Carnot no diagrama $TS$, composto de duas
isotermas e duas adiabáticas:
\begin{figure}[H]
    \centering
    \begin{tikzpicture}
        \draw[->, thick]
            (0,0) -- (5,0) node[below] {$S$};
        \draw[->, thick]
            (0,0) -- (0,4) node[left] {$T$};
        \draw
            (1.5,0.2) -- (1.5,-0.2) node[below] {$S_\text{min}$};
        \draw
            (3.5,0.2) -- (3.5,-0.2) node[below] {$S_\text{max}$};
        \draw
            (0.2,1.2) -- (-0.2,1.2) node[left] {$T_C$};
        \draw
            (0.2,2.8) -- (-0.2,2.8) node[left] {$T_H$};
        \fill[lightgray]
            (1.5,1.2) -- (1.5,2.8) -- (3.5,2.8) -- (3.5,1.2) -- cycle;
        \draw[ultra thick,postaction={mid arrow}]
            (1.5,1.2) -- (1.5,2.8);
        \draw[ultra thick,postaction={mid arrow}]
            (1.5,2.8) -- (3.5,2.8);
        \draw[ultra thick,postaction={mid arrow}]
            (3.5,2.8) -- (3.5,1.2);
        \draw[ultra thick,postaction={mid arrow}]
            (3.5,1.2) -- (1.5,1.2);
    \end{tikzpicture}
\end{figure}
A entropia faz o papel análogo a uma coordenada de trabalho conjugada à
temperatura termodinâmica. A expressão diferencial para a Primeira Lei então
fica com um formato simétrico envolvendo a entropia:
$$dU=TdS+y_1dX_1+\cdots y_ndX_n,$$
onde o termo $\delta Q=TdS$. Note também a extensividade de $S$, que garante que
quando escrevemos em termos de variáveis extensivas $X$, vale que
$$S(\lambda X)=\lambda S(X)$$
para qualquer fator de escala $\lambda$. Extensividade também finalmente nos
proporciona a relação de Gibbs-Duhem para qualquer sistema termodinâmico
$$SdT+X_1dy_1+\cdots X_ndy_n=0.$$

\subsection{Caracterizando o Equilíbrio}

A formulação entrópica da Segunda Lei nos permite juntar a noção de equilíbrio
térmico com o equilíbrio das outras variáveis do nosso sistema. De acordo com a
Segunda Lei, todas as transformações possíveis para um sistema isolado são
aquelas que acarretam em um aumento de entropia. Ou seja, o \emph{equilíbrio
termodinâmico} de um sistema é caracterizado por um \emph{máximo} de entropia.
Para um sistema isolado de energia interna total $U$ e variáveis de estado $a_1,
\dots,a_n$ que podem variar, no equilíbrio estas tomam o valor tal que a
entropia total do sistema $S$ satisfaz
$$S(U,a'_1,\dots,a'_n)\leq S(U,a_1,\dots,a_n).$$
Com uma simples manipulação que utiliza o fato de que a temperatura
termodinâmica deve ser positiva, podemos também caracterizar o equilíbrio de um
sistema que pode receber trabalho de um meio externo sem receber calor através
de um \emph{mínimo} de energia interna, ou seja
$$U(S,a'_1,\dots,a'_n)\geq U(S,a_1,\dots,a_n).$$

\subsection{Problemas}

\begin{enumerate}
    \item
        (SOIF 2002) Considere uma geladeira ligada dentro de um quarto isolado a
        uma temperatura $T_0$. Sejam $T_a,T_b$, e $T_c$ as temperaturas do
        quarto após um longo tempo com a geladeira respectivamente
        \begin{enumerate}
            \item
                fechada e vazia,
            \item
                fechada e cheia de comida,
            \item
                aberta.
        \end{enumerate}
        Dê a ordem das temperaturas $T_0,T_a,T_b$ e $T_c$.
        \answer{$T_0<T_a<T_b<T_c$}

    \item
        Abaixo temos a representação do ciclo de Otto num diagrama $pV$:
        \begin{figure}[H]
            \centering
            \begin{tikzpicture}
                \draw[very thick,->]
                    (0,0) -- (4,0) node[below] {$V$};
                \draw[very thick,->]
                    (0,0) -- (0,3) node[left] {$p$};

                \draw[fill=lightgray,ultra thick]
                    (2.4,0.6) 
                    plot[samples=15,domain=2.4:0.8]
                        (\x,{0.6*(2.4/\x)^0.7}) node[left] {2}
                    -- (0.8,2.7) node[left] {3} 
                    plot[samples=15,domain=0.8:2.4]
                        (\x,{2.7*(0.8/\x)^0.7}) node[right] {4} 
                    -- (2.4,0.6);
                \draw[ultra thick]
                    (0.8,0.5) node[below] {0}
                    -- (2.4,0.5) node[below] {1}
                    -- (2.4,0.6) -- (0.8,0.6);

                \path[ultra thick,postaction={mid arrow}]
                    (0.8,0.6) -- (2.4,0.6);
                \path[ultra thick,postaction={mid arrow}]
                    plot[samples=15,domain=2.4:0.8] (\x,{0.6*(2.4/\x)^0.7});
                \path[ultra thick,postaction={mid arrow}]
                    (0.8,{0.6*(2.4/0.8)^0.7}) -- (0.8,2.7);
                \path[ultra thick,postaction={mid arrow}]
                    plot[samples=15,domain=0.8:2.4] (\x,{2.7*(0.8/\x)^0.7});
                \path[ultra thick,postaction={mid arrow}]
                    (2.4,{2.7*(0.8/2.4)^0.7}) -- (2.4,0.5);
                \path[ultra thick,postaction={mid arrow}]
                    (2.4,0.5) -- (0.8,0.5);
            \end{tikzpicture}
        \end{figure}
        Este ciclo é utilizado em motores a combustão.
        \begin{itemize}
            \item 
                $0\rightarrow1$ o motor recebe uma quantidade de ar da atmosfera
                e aquece-o isobaricamente até a temperatura de trabalho do motor
            \item
                $1\rightarrow2$ injeta-se uma pequena quantidade de combustível
                no cilíndro do motor e realizamos uma compressão adiabática por
                um fator $r>1$.
            \item
                $2\rightarrow3$ uma faísca causa a ignição do combustível,
                gerando um aquecimento isocórico do ar
            \item
                $3\rightarrow4$ permitimos que o ar expanda adiabaticamente até
                o volume antes da compressão
            \item
                $4\rightarrow0$ expelimos o ar sujo de fuligem para a atmosfera,
                e aproximamamos este processo por um resfriamento isocórico até
                a pressão atmosférica e depois isobárico de volta à temperatura 
                ambiente
        \end{itemize}
        \begin{enumerate}
            \item
                Encontre a eficiência $\eta_\text{Otto}$ do ciclo de Otto em
                termos de $r$ e do coeficiente de Poisson $\gamma$ do ar.
                \answer{$\eta_\text{Otto}=1-\frac{1}{r^{\gamma-1}}$}
        \end{enumerate}
        Podemos no entanto aumentar a eficiência do ciclo permitindo que a
        expansão passe um pouco do volume inicial da compressão, expandindo por
        um fator $s>r$: 
        \begin{figure}[H]
            \centering
            \begin{tikzpicture}
                \draw[very thick,->]
                    (0,0) -- (4,0) node[below] {$V$};
                \draw[very thick,->]
                    (0,0) -- (0,3) node[left] {$p$};

                \fill[lightgray]
                    (2.4,0.6) 
                    plot[samples=15,domain=2.4:0.8] 
                        (\x,{0.6*(2.4/\x)^0.7})
                    -- (0.8,2.7) 
                    plot[samples=15,domain=0.8:2.4]
                        (\x,{2.7*(0.8/\x)^0.7})
                    -- (2.4,0.6);
                \fill[gray]
                    (2.4,0.6) -- (2.4,{2.7*(0.8/2.4)^0.7})
                    plot[samples=6,domain=2.4:3.5]
                        (\x,{2.7*(0.8/\x)^0.7})
                    -- (3.5,0.6) -- (2.4,0.6);
                \draw[ultra thick]
                    (2.4,0.6) 
                    plot[samples=15,domain=2.4:0.8]
                        (\x,{0.6*(2.4/\x)^0.7}) node[left] {2}
                    -- (0.8,2.7) node[left] {3} 
                    plot[samples=15,domain=0.8:3.5]
                        (\x,{2.7*(0.8/\x)^0.7}) node[right] {4} 
                    -- (3.5,0.6) -- (2.4,0.6);
                \draw[ultra thick]
                    (0.8,0.6) -- (2.4,0.6);
                \node[below]
                    at (2.4,0.5) {1};
                \draw[ultra thick]
                    (3.5,0.6)-- (3.5,0.5) -- (0.8,0.5) node[below] {0};

                \path[ultra thick,postaction={mid arrow}]
                    (0.8,0.6) -- (2.4,0.6);
                \path[ultra thick,postaction={mid arrow}]
                    plot[samples=15,domain=2.4:0.8] (\x,{0.6*(2.4/\x)^0.7});
                \path[ultra thick,postaction={mid arrow}]
                    (0.8,{0.6*(2.4/0.8)^0.7}) -- (0.8,2.7);
                \path[ultra thick,postaction={mid arrow}]
                    plot[samples=15,domain=0.8:3.5] (\x,{2.7*(0.8/\x)^0.7});
                \path[ultra thick,postaction={mid arrow}]
                    (3.5,{2.7*(0.8/3.5)^0.7}) -- (3.5,0.5);
                \path[ultra thick,postaction={mid arrow}]
                    (3.5,0.5) -- (0.8,0.5);
            \end{tikzpicture}
        \end{figure}
        Este é chamado ciclo de Miller.
        \begin{enumerate}
            \setcounter{enumii}{1}
            \item
                Mostre que mantendo a temperatura de trabalho e $r$ fixos a
                eficiência do ciclo de Miller aumenta com $s$.
                \answer{a área do ciclo aumenta sem mudar o calor absorvido}        
            \item
                Para uma temperatura de trabalho fixo existe um valor de $s$
                máximo. Neste caso, o ciclo é chamado de ciclo de Atkinson.
                Encontre a eficiência $\eta_\text{Atkinson}$ em termos de $r$,
                $\gamma$ e do valor máximo possível para $s$.
                \answer{
                    $\eta_\text{Atkinson}=1-\gamma\frac{s-r}{s^\gamma-r^\gamma}$
                }
            \item
                O ciclo de Atkinson, apesar de mais eficiente, apresenta uma
                desvantagem em relação ao ciclo de Otto. Discuta.
                \answer{
                    o volume máximo do cilíndro precisa ser maior para obtermos
                    o mesmo trabalho para uma mesma quantidade de combustível
                }
        \end{enumerate}

    \item
        (SOIF 2002) No diagrama $pV$ abaixo estão esquematizados dois ciclos
        para um gás ideal. O rendimento do ciclo $1\rightarrow2\rightarrow4
        \rightarrow1$ é $\eta_1$, enquanto o rendimento do ciclo $2\rightarrow3
        \rightarrow4\rightarrow2$ é $\eta_2$. Sabendo que todos os processos
        envolvidos são segmentos de reta de coeficiente angular positivo,
        encontre o rendimento $\eta$ do ciclo $1\rightarrow2\rightarrow3
        \rightarrow4 \rightarrow1$.
        \begin{figure}[H]
            \centering
            \begin{tikzpicture}[scale=1.2]
                \draw[very thick,->]
                    (0,0) -- (3,0) node[below] {$V$};
                \draw[very thick,->]
                    (0,0) -- (0,3) node[left] {$p$};
                \draw[fill=lightgray,ultra thick]
                    (0.5,0.5) node[below left] {4}
                    -- (2.3,0.7) node[below right] {3}
                    -- (2.7,2.6) node[above right] {2}
                    -- (0.8, 2.1) node[above left] {1} -- cycle;
                \draw[ultra thick]
                    (0.5,0.5) -- (2.7,2.6);
            \end{tikzpicture}
        \end{figure}
        \answer{$\eta=\eta_1+\eta_2-\eta_1\eta_2$}

    \item
        (Irodov) Um ciclo termodinâmico para um gás ideal consiste em uma
        isoterma, um processo politrópico e uma adiabata, com o processo
        isotérmico ocorrendo à temperatura máxima do ciclo. Encontre a
        eficiência $\eta$ deste ciclo sabendo que a a temperatura máxima é
        $\alpha$ vezes a temperatura mínima do ciclo.
        \answer{$\eta=1-\frac{\alpha-1}{\alpha\log\alpha}$}

    \item
        O nome de temperatura termodinâmica e a utilização da letra $T$ para a
        temperatura na equação de estado do gás ideal não é coincidência.
        Partindo das equações
        $$pV=NRT,\quad U=\frac{NRT}{\gamma-1}$$
        \begin{enumerate}
            \item
                Construa o ciclo de Carnot através de duas isotermas e duas
                adiabáticas para concluir que $T$ é uma escala de temperatura
                termodinâmica.
            \item
                Escreva uma expressão para a entropia do gás ideal em termos da
                energia interna $U$, volume $V$, número de moléculas $N$,
                coeficiente de Poisson $\gamma$ e uma constante arbitrária.
                \answer{
                    $S=\frac{NR}{\gamma-1}
                    \log\frac{UV^{\gamma-1}}{\Phi N^\gamma}$
                }
        \end{enumerate}

    \item
        Mostre que se a pressão de um sistema pode ser escrita $p=Tf(V)$ então a
        energia interna $U$ depende apenas da temperatura $T$. Conclua que a
        equação para a energia interna do gás ideal segue da equação de estado
        para a pressão e da suposição que $C_V$ é independente da temperatura.
        \answer{$\pd{U}{V}{T}=T\pd{p}{T}{V}-p$}

    \item
        O ciclo de Stirling para um gás ideal é definido 
        \begin{itemize}
            \item 
                $1\rightarrow2$ compressão isotérmica por um fator $r$
            \item
                $2\rightarrow3$ aquecimento isocórico até $T_H$
            \item
                $3\rightarrow4$ expansão isotérmica
            \item
                $4\rightarrow1$ resfriamento isocórico até $T_C$
        \end{itemize}
        \begin{enumerate}
            \item
                Desenhe o ciclo num diagrama $pV$.
                \answer{
                    \begin{tikzpicture}
                        \draw[very thick,->]
                            (0,0) -- (3,0) node[right] {$V$};
                        \draw[very thick,->]
                            (0,0) -- (0,2) node[above] {$p$};
                        
                        \draw
                            (0.8,0.1) -- (0.8,-0.1) node[below] {$V_0$};
                        \draw
                            (2.5,0.1) -- (2.5,-0.1) node[below] {$rV_0$};
                        \draw
                            (0.1,0.7) -- (-0.1,0.7) node[left] {$\alpha T_C$};
                        \draw
                            (0.1,1.7) -- (-0.1,1.7) node[left] {$\alpha T_H$};

                        \draw[ultra thick,fill=lightgray]
                            (2.5,{0.7*0.8/2.5}) node[right] {1}
                            plot[domain=2.5:0.8,samples=10]
                                (\x,{0.7*0.8/\x}) node[left] {2}
                            -- (0.8,1.7) node[left] {3}
                            plot[domain=0.8:2.5,samples=10]
                                (\x,{1.7*0.8/\x}) node[above] {4}
                            -- (2.5,{0.7*0.8/2.5});

                        \path[ultra thick,postaction={mid arrow}]
                            plot[domain=2.5:0.8,samples=10] (\x,{0.7*0.8/\x});
                        \path[ultra thick,postaction={mid arrow}]
                            (0.8,0.7) -- (0.8,1.7);
                        \path[ultra thick,postaction={mid arrow}]
                            plot[domain=0.8:2.5,samples=10] (\x,{1.7*0.8/\x});
                        \path[ultra thick,postaction={mid arrow}]
                            (2.5,{1.7*0.8/2.5}) -- (2.5,{0.7*0.8/2.5});
                    \end{tikzpicture}
                }
            \item
                Desenhe o ciclo num diagrama $TS$.
                \answer{
                    \begin{tikzpicture}
                        \draw[very thick,->]
                            (0,0) -- (3,0) node[right] {$S$};
                        \draw[very thick,->]
                            (0,0) -- (0,2) node[above] {$T$};
                        
                        \draw
                            (0.8,0.1) -- (0.8,-0.1) node[below] {$0$};
                        \draw
                            (2.3,0.1) -- (2.3,-0.1) node[below] {$NR\log r$};
                        \draw
                            (0.1,0.7) -- (-0.1,0.7) node[left] {$T_C$};
                        \draw
                            (0.1,1.7) -- (-0.1,1.7) node[left] {$T_H$};

                        \draw[ultra thick,fill=lightgray]
                            (0.8,0.7)
                            plot[domain=0.8:1.2,samples=10]
                                (\x,{0.7*(1.7/0.7)^((\x-0.8)/0.4)})
                                node[left] {3}
                            -- (2.7,1.7) node[right] {4}
                            plot[domain=2.7:2.3,samples=10]
                                (\x,{0.7*(1.7/0.7)^((\x-2.3)/0.4)})
                                node[right] {1} 
                            -- (0.8,0.7) node[left] {2};

                        \path[ultra thick,postaction={mid arrow}]
                            plot[domain=0.8:1.2,samples=10]
                                (\x,{0.7*(1.7/0.7)^((\x-0.8)/0.4)});
                        \path[ultra thick,postaction={mid arrow}]
                            (2.3,0.7) -- (0.8,0.7);
                        \path[ultra thick,postaction={mid arrow}]
                            plot[domain=2.7:2.3,samples=10]
                                (\x,{0.7*(1.7/0.7)^((\x-2.3)/0.4)});
                        \path[ultra thick,postaction={mid arrow}]
                            (1.2,1.7) -- (2.7,1.7);
                    \end{tikzpicture}
                }
            \item
                Encontre a eficiência $\eta$ do ciclo supondo que ele ocorre num
                gás ideal de coeficiente de Poisson $\gamma$ e mostre que é
                menor do que a eficiência de Carnot operando entre $T_C$ e
                $T_H$.
                \answer{
                    $\eta=1-\frac{T_C+\Delta T}{T_H+\Delta T},
                    \quad\Delta T=\frac{T_H-T_C}{(\gamma-1)\log r}$
                }
        \end{enumerate}
        Para os gráficos introduza constantes para que marcações nos eixos façam
        sentido.

    \item
        (SOIF 2022) Um físico passará o inverno isolado dentro de casa e
        projetará um sistema de aquecimento para garantir o seu conforto. O
        sistema visa manter a temperatura no interior da casa em $T_C=278
        \,\mathrm K$, enquanto a temperatura do ambiente externo é $T_F=263
        \,\mathrm K$. O sistema dispõe de uma lareira que pode servir de
        reservatório térmico de temperatura $T_Q=600\,\mathrm K$. Temos o
        objetivo de transferir uma quantidade de calor $Q$ para dentro da casa
        com o menor custo energético possível. Assumindo que podemos transferir
        calor da lareira tanto do interior quanto para o exterior da casa, qual
        a menor quantidade de calor $Q'$ que precisa ser extraida da lareira
        para que isso seja possível?
        \answer{$Q'=\frac{T_Q(T_C-T_F)}{T_C(T_Q-T_F)}Q=0{,}096Q$}

    \item
        Considere dois objetos de capacidade térmica $C_1$ e $C_2$ com
        respectivas temperaturas $T_1$ e $T_2$. Utilizando apenas um conjunto de
        máquinas térmicas sem fontes de calor externas, qual a menor temperatura
        de equilíbrio possível para estes objetos? Comente sobre o caso $C_2\ll
        C_1$.
        \answer{
            $T_0=\left(T_1^{C_1}T_2^{C_2}\right)^\frac{1}{C_1+C_2}
            \approx T_1+\frac{C_2}{C_1}T_1\log\frac{T_2}{T_1}$
        }

    \item
        Considere um cilíndro adiabático vertical de área de seção transfersal
        $A$. No interior deste há uma certa quantidade de fluido. Coloque um
        pistão adiabático de massa $m$ e área também $A$ que se move sem atrito
        pelo cilíndro. Dado que a gravidade local é $g$ e a pressão atmosférica
        é $p_0$, encontre a pressão do gás no equilíbrio termodinâmico.
        \answer{$p=p_0+\frac{mg}{A}$}
\end{enumerate}

\section{Transições de Fase}

\subsection{Potenciais Termodinâmicos}

Da maneira como foram formuladas as leis da termodinâmica, vemos que as
variáveis naturais para se trabalhar com o assunto são a energia interna, a
entropia, e as coordenadas de trabalho do sistema. No entanto, em muitos
contextos teóricos e experimentais nosso interesse é no que acontece quando
especificamos alguns parâmetros intensivos ao invés dos extensivos naturais.
Especificar a pressão de um gás ao invés de seu volume. A tensão num elástico ao
invés de sua distensão. A temperatura de um objeto ao invés de sua entropia. 

Fisicamente, há uma maneira muito simples de implementar este tipo de vínculo.
Suponha que a primeira lei da termodinâmica para um sistema físico pode ser
escrita
$$dU=p_1dQ_1+\cdots+p_ndQ_n+y_1dX_1+\cdots+y_ndX_n,$$
onde chamamos $p,Q$ variáveis que queremos controlar o lado intensivo, enquanto
$y,X$ as que queremos controlar o lado extensivo. Ao invés de tratarmos o
sistema como isolado, acomplamos o mesmo a algum tipo de reservatório que pode
absorver as variáveis extensivas conjugadas às intensivas que estamos tentando
fixar, sendo este grande o suficiente de forma que possamos aproximar sua
energia interna até primeira ordem:
$$U_\text{res}(Q_1,\dots,Q_m)
\approx U^{(0)}+p^{(0)}_1(Q_1-Q^{(0)}_1)+\cdots+p^{(0)}_m(Q_m-Q^{(0)}_m).$$

Realizamos o acoplamento olhando então para a energia total do conjunto,
$$U_\text{tot}=U(Q_1,\dots,Q_m,X_1,\dots,X_n)
+U_\text{res}(Q^{(0)}_1-Q_1,\dots,Q_m)$$
e deixamos o sistema encontrar seu equilíbrio isolado do resto do universo. Isso
é equivalente a encontrar os $Q$ que minimizam a expressão
$$V=U-p^{(0)}_1Q_1-\cdots-p^{(0)}_mQ_m,$$
levando-nos a concluir que as variáveis intensivas devem ter o mesmo valor que
têm no reservatório.
$$p_1=p^{(0)}_1,\dots,p_m=p^{(0)}_m.$$
Podemos resumir essa construção analisando a função $V(p_1,\dots,p_m,X_1,\dots,
X_n)$ definida como a transformação de Legendre de $U$ nas variáveis de
interesse, e recebe o nome de um \emph{potencial termodinâmico}. Como nestes
casos estamos normalmente nos preocupando apenas com a configuração das
variáveis $X$, a caracterização do equilíbrio como ponto que minimiza $V$
continua sendo válida. Note também que dada a extensividade, se fizermos a
transformação utilizando todas as variáveis extensivas, a função que sobra é
identicamente nula e perdemos qualquer informação sobre o que é sistema e o que
é reservatório. Precisamos pelo menos de uma grandeza extensiva de referência
nos nossos potenciais termodinâmicos. 

Alguns casos específicos dessa construção recebem nomes. Para qualquer sistema
podemos realizar a transformação $F=U-TS$, que recebe o nome de \emph{energia
livre de Helmholtz}. Quando lidamos com gases e outros fluidos que possuem
pressão e volume, definimos também a \emph{entalpia} $H=U+pV$ e a \emph{energia
livre de Gibbs} $G=U-TS+pV$. Ainda para um fluido, tomando a segunda derivada
dessas funções em diferentes ordens nos permite extrair as \emph{relações de
Maxwell}:
\begin{alignat*}{3}
    &&U_{,SV}&=\pd{T}{V}{S}&&=-\pd{p}{S}{V}\\
    &&H_{,Sp}&=\pd{T}{p}{S}&&=\pd{V}{S}{p}\\
    &&-F_{,TV}&=\pd{S}{V}{T}&&=\pd{P}{T}{V}\\
    &&-G_{,Tp}&=\pd{S}{p}{T}&&=-\pd{V}{T}{p}
\end{alignat*}

\subsection{Estabilidade}

Vamos considerar um sistema termodinâmico em que escrevemos sua entropia em
termos das variáveis extensivas $S(X)$. De maneira geral visualizamos essa
situação como as variáveis extensivas uniformemente distribuidas ao longo da
extensão do corpo, porém na realidade essas grandezes flutuam ao longo do tempo,
de forma que a princípio podemos observar uma parcela $\Delta$ em proporções
diferentes do resto do corpo $X-\Delta$:
\begin{figure}[H]
    \centering
    \begin{tikzpicture}
        \begin{scope}[xshift=-2.5cm]
            \draw[very thick,fill=lightgray]
                (0,0) -- (3,0) -- (3,3) -- (0,3) -- cycle;
            \node
                at (1.5,1.5) {$X$};
        \end{scope}
        \draw[ultra thick,->]
            (1,1.5) -- (2,1.5);
        \begin{scope}[xshift=2.5cm]
            \draw[very thick,fill=lightgray]
                (0,0) -- (3,0) -- (3,3) -- (0,3) -- cycle;
            \draw[thick,fill=gray,use Hobby shortcut,closed=true] 
                (0.7,1.5) .. (1.2,1.8) .. (1.1,2.0) .. (1.3,2.2) .. (0.7,2.3);
            \node
                at (0.85,2) {$\Delta$};
            \node
                at (2,0.8) {$X-\Delta$};
        \end{scope}
    \end{tikzpicture}
\end{figure}
Nessa situação a entropia varia de $S(X)$ para $S(\Delta)+S(X-\Delta)$, e
portanto será proibida pela Segunda Lei quando isso corresponder a uma
diminuição de entropia. Ou seja, concluimos que os pontos de equilíbrio
possíveis para nosso sistema são apenas aqueles $X$ onde
$$S(X)\geq S(X-\Delta)+S(\Delta),$$
para toda variação $\Delta$, e a entropia é uma função \emph{côncava} das
variáveis de estado extensivas. Do contrário, observariamos rapidamente
fenômenos de \emph{nucleação} como descritos acima. O caso de igualdade precisa
ser incluido devido a extensividade, e só é encontrado no caso em que $\Delta
\sim X$.

Supondo agora apenas que a temperatura termodinâmica do nosso sistema é
positiva, uma simples manipulação análoga à que foi feita na condição de
equiilíbrio nos leva também a concluir a \emph{convexidade} da energia interna
do sistema em termos das variáveis extensivas:
$$U(X)\leq U(X-\Delta)+U(\Delta).$$
Lembrando da expressão da primeira lei para processos contínuos, podemos também
escrever uma relação entre os diferenciais das variáveis de estado que precisa
ser satisfeita para todo processo possível em qualquer estado de equilíbrio
estável:
$$dy_1dX_1+\cdots+dy_ndX_n\geq0,$$
onde a igualdade segue da relação de Gibbs-Duhem em uma transformação de escala
dada por $dX\sim X$. Essa condição de estabilidade local pode ser satisfeita até
por estados que violam a estabilidade global, levando à \textit{metaestabilidade
} de alguns estados, que são resistentes a pequenas flutuações mas que estão
sujeitos a nucleação para choques suficientemente grandes.

Para outros potenciais termodinâmicos, novamente nos interessamos apenas na
variação das variáveis extensivas que sobraram, e é fácil de ver que a
convexidade nestas é preservada pela transformação de Legendre.

\subsection{Fases da Matéria}

Suponha que certo sistema possui entropia
$$S(X_1,\dots,X_n,N),$$
onde denotamos uma variável extensiva $N$ específica que vamos utilizar para
controlar o seu tamanho. Podemos definir então a \emph{entropia específica} 
$$s(x_1,\dots,x_n)=\frac{S(Nx_1,\dots,Nx_n,N)}{N},$$
que por extensividade está bem definida independente do valor de $N$ que
escolhermos, e as variáveis intensivas $x_1,\dots,x_n$ são chamadas \emph{
específicas}. Neste caso, ganhamos um novo sentido para a questão da
estabilidade. Um estado $(Nx,N)$ será estável se, e somente se para quaisquer
pares de pontos $x^{(0)},x^{(1)}$ tais que existe $0<t<1$ de forma que $x=(1-t)
x^{(0)}+ tx^{(1)}$, vale que
$$s(x)>(1-t)s(x^{(0)})+ts(x^{(1)}).$$
Caso contrário, o sistema se decompõe em duas \emph{fases} de estados $((1-t)
x^{(0)},(1-t)N)$ e $(tx^{(1)},tN)$. Geometricamente, vemos que um estado $x$ é
estável quando o gráfico de $s$ fica abaixo do plano tangente ao mesmo em $x$.
No gráfico abaixo vemos um exemplo de região instável no caso de uma única
variável $x$.
\begin{figure}[H]
    \centering
    \begin{tikzpicture}
        \draw[very thick,->] 
            (0,0) -- (8,0) node[right] {$x$};
        \draw[very thick,->]
            (0,0) -- (0,3) node[left] {$s$};
            \begin{luacode}
                function f(x, m)
                    t = math.pi * (x - 2.5) / (5.5 - 2.5)
                    return 
                        -1.2 * math.log(x) 
                        + 0.5 * math.sin(t)^2
                        + m * x
                end

                function d(m)
                    local x0 = minimize(f, {2.5, 3.5}, m)
                    local x1 = minimize(f, {4.5, 5.5}, m)
                    return f(x1, m) - f(x0, m)
                end

                m = solve(d, {0.0, 0.5})
                x0 = minimize(f, {2.5, 3.5}, m)
                x1 = minimize(f, {4.5, 5.5}, m)

                tex.print(
                    "\\draw[fill=lightgray,thick,dashed]",
                    "(", x0, ",0) -- (", x0,",", -f(x0, 0), ")", 
                    "-- (", x1, ",", -f(x1, 0), ") -- (", x1, ",0);"
                )

                function g(x) 
                    t = math.pi * (x - 2.5) / (5.5 - 2.5)
                    return -1.2 / x^2 - math.cos(2 * t) / (5.5 - 2.5)^2
                end

                y0 = solve(g, {2.5, 4.0})
                y1 = solve(g, {4.0, 5.5})

                tex.print(
                    "\\draw[ultra thick,samples=10]",
                    "plot[domain=1:2.5]",
                    "(\\x,{1.2*ln(\\x)})",
                    "plot[domain=2.5:", y0, "]",
                    "(\\x,{1.2*ln(\\x)",
                    "-0.5*pow(sin((\\x-2.5)*180/(5.5-2.5)),2)});"
                )
                tex.print(
                    "\\draw[ultra thick,dashed,domain=", 
                    y0, ":", y1, ",samples=10]",
                    "plot (\\x,{1.2*ln(\\x)",
                    "-0.5*pow(sin((\\x-2.5)*180/(5.5-2.5)),2)});"
                )
                tex.print(
                    "\\draw[ultra thick,samples=10]",
                    "plot[domain=", y1, ":5.5]",
                    "(\\x,{1.2*ln(\\x)",
                    "-0.5*pow(sin((\\x-2.5)*180/(5.5-2.5)),2)})",
                    "plot[domain=5.5:7,samples=5]",
                    "(\\x,{1.2*ln(\\x)});"
                )
            \end{luacode}
    \end{tikzpicture}
\end{figure}
A região metaestável foi diferenciada da totalmente instável pela linha
pontilhada. Qualquer estado preparado com $x$ na região instável gera uma
coexistência de fases na sua fronteira, e como a construção vem de retas
tangentes vemos que todas as fases da matéria que coexistem devem compartilhar o
valor de todos os coeficientes de trabalho - estão em equilíbrio termodinâmico
umas com as outras. Assim, quando analisamos transições de fase é natural
utilizar um potencial termodinâmico que deixa esses coeficientes intensivos em
primeiro plano. Escrevendo a Primeira Lei como
$$dU=y_1dX_1+\cdots y_ndX_n+\mu dN,$$
olhamos para o potencial $V=U-y_1X_1-\cdots y_nX_n$, genericamente chamado de
energia livre, que por extensividade é simplesmente dado por
$$V=\mu N.$$
Assim, identificamos pontos nessa fronteira encontrando processos contínuos $i
\rightarrow f$ tais que $y_1^{(i)}=y_1^{(f)},\dots,y_n^{(i)}=y_n^{(f)}$ e
$$\mu^{(i)}=\mu^{(f)}\Rightarrow\Delta V=-\int_i^fX_1dy_1+\cdots+X_ndy_n=0.$$
Apesar de que os valores dos coeficientes de trabalho devem ser o mesmo de ambos
os lados do processo, podemos ter mudanças nas propriedades das coordenadas de
trabalho. Mais precisamente, observamos possíveis diferenças nos valores ou nas
derivadas destes. A \emph{ordem} de uma transição de fase é o nome que damos
para a menor ordem de derivada da energia livre que é diferente antes e depois
da transição. Uma simples contagem de equações e variáveis livres também nos
mostra que o espaço dos coeficientes de trabalho onde $k$ fases podem coexistir
é $(n+1-k)$--dimensional.

\subsection{O Gás de van der Waals}

No caso de um fluido simples que possui energia interna, volume e número de
moléculas, o parâmetro $N$ utilizado normalmente é o número de moléculas e o
coeficiente $\mu$ recebe o nome de \emph{potencial químico}, e o potencial
termodinâmico de interesse é a energia livre de Gibbs. Considere portanto o gás
ideal
$$S=\frac{NR}{\gamma-1}\log\frac{UV^{\gamma-1}}{\Phi N^\gamma}.$$
Podemos incrementar o modelo para melhor representar gases reais através das
modificações
\begin{align*}
    U&\mapsto U+a\frac{N^2}{V}\\
    V&\mapsto V-Nb
\end{align*}
que correspondem respectivamente a uma força atrativa que tende a manter a
coesão do fluido, e a um limite intrínseco a quanto podemos comprimí-lo. 
As equações de estado correspondentes à lei dos gases ideais então imediatamente
tornam-se
$$U=\frac{NRT}{\gamma-1}-a\frac{N^2}{V},$$
$$\left(p+a\frac{N^2}{V^2}\right)\left(\frac{V}{N}-b\right)=RT.$$
Mas ao fazermos o gráfico de algumas das isotermas imediatamente nos deparamos
com um problema:
\begin{figure}[H]
    \centering
    \begin{tikzpicture}
        \draw[very thick,->]
            (0,0) -- (4,0) node[below] {$V$};
        \draw[very thick,->]
            (0,0) -- (0,3) node[left] {$p$};
        \begin{luacode}
            t = {0.86, 1.00, 1.1}
            f = {"black", "darkgray", "gray"}

            function p(v, t, c)
                return 1.5 * (8 * t / (3 * v - 1) - 3 / (v * v)) - c
            end
            
            for i = 1, 3 do
                tex.print(
                    "\\draw[ultra thick,samples=20,", f[i], "]",
                    "plot[domain=", solve(p, {0.4, 1}, t[i], 2.8), ":1]",
                    "(\\x,{1.5*(8*", t[i], "/(3*\\x-1)-3/(\\x*\\x))})",
                    "plot[domain=1:3.5]",
                    "(\\x,{1.5*(8*", t[i], "/(3*\\x-1)-3/(\\x*\\x))});"
                )
            end
        \end{luacode}
    \end{tikzpicture}
\end{figure}
\noindent Em uma delas temos uma região onde a pressão aumenta com o volume à
temperatura constante, uma clara violação das condições de estabilidade.

Ao invés de descartar o modelo, vamos tentar caracterizar a transição de fase
que está implícita nas regiões instáveis. Primeiro vemos que o comportamento só
aparece em isotermas de temperatura
$$T<T_c=\frac{8a}{27Rb},$$
chamada temperatura \emph{crítica} do modelo, que é a isoterma específica
apresentando um ponto de inflexão horizontal em
$$V=Nv_c=3Nb,\quad p=p_c=\frac{a}{27b^2}.$$
Podemos então colocar a equação de estado em termos do que chamamos de variáveis
\emph{reduzidas} para o gás de van der Waals,
$$T_r=T/T_c,\quad v_r=V/Nv_c,\quad p_r=p/p_c,$$
$$p_r=\frac{8T_r}{3v_r-1}-\frac{3}{v_r^2}.$$
A variação da energia livre de Gibbs ao longo de uma isoterma pode então ser
escrita
$$dG=Vdp\sim v_rdp_r,$$
e impondo que esta seja zero entre ambas as fronteiras de dada isoterma obtemos
a \emph{construção de Maxwell} para o gás de van der Waals
\begin{figure}[H]
    \centering
    \begin{tikzpicture}
        \draw[very thick,->]
            (0,0) -- (5,0) node[below] {$v_r$};
        \draw[very thick,->]
            (0,0) -- (0,3.5) node[left] {$p_r$};
        \begin{luacode}
            t = 0.86

            function p(v, k, p0)
                return k * (8 * t / (3 * v - 1) - 3 / (v * v)) - p0
            end

            vlmin = 0.5
            vlmax = minimize(p, {vlmin, 1}, 1, 0)

            vgmax = 5.0
            vgmin = minimize(p, {1, vgmax}, -1, 0)

            function g(v)
                return 
                    4 * t * (1 / (3 * v - 1) - math.log(3 * v - 1)) / 9
                    - 1 / v
            end

            function dg(p0)
                local vl = solve(p, {vlmin, vlmax}, 1, p0)
                local vg = solve(p, {vgmin, vgmax}, 1, p0)
                return g(vg) - g(vl)
            end

            p0 = solve(dg, {0.13, 0.63})
            vl = solve(p, {vlmin, vlmax}, 1, p0)
            vg = solve(p, {vgmin, vgmax}, 1, p0)
            vm = solve(p, {vlmax, vgmin}, 1, p0)

            tex.print(
                "\\draw (0.1,", 3 * p0, ") -- (-0.1,", 3 * p0, ")",
                "node[left] {$p^{(0)}$};"
            )
            tex.print(
                "\\draw (", vl, ",0.1) -- (", vl, ",-0.1)",
                "node[below] {$v^{(i)}$};"
            )
            tex.print(
                "\\draw (", vg, ",0.1) -- (", vg, ",-0.1)",
                "node[below] {$v^{(f)}$};"
            )

            tex.print(
                "\\draw[very thick,dashed,samples=10,fill=lightgray]",
                "plot[domain=", vl, ":", vm, "]",
                "(\\x,{3*(8*", t, "/(3*\\x-1)-3/(\\x*\\x))});"
            )

            tex.print(
                "\\draw[very thick,dashed,samples=10,fill=gray]",
                "plot[domain=", vm, ":", vg, "]",
                "(\\x,{3*(8*", t, "/(3*\\x-1)-3/(\\x*\\x))});"
            )

            tex.print(
                "\\draw[ultra thick,samples=10]",
                "plot[domain=", solve(p, {0.4, 1}, 1, 3.2 / 3), ":", vl, "]",
                "(\\x,{3*(8*", t, "/(3*\\x-1)-3/(\\x*\\x))})",
                "-- plot[domain=", vg, ":4.5]",
                "(\\x,{3*(8*", t, "/(3*\\x-1)-3/(\\x*\\x))});"
            )
        \end{luacode}
    \end{tikzpicture}
\end{figure}
\noindent que corresponde à igualdade das áreas sombreadas no gráfico. Vemos que
a transição se dá à pressão $p=p^{(0)}p_c$ entre uma fase densa incompressível 
de volume $Nv_l=Nv^{(i)}v_c$ até uma fase rarefeita compressível de volume $Nv_g
=Nv^{(f)}v_c$. Devido à expansão na transição, vemos a partir da conservação de
energia que o sistema precisa absorver calor $Q=NL>0$. A instabilidade no gas de
van der Waals nos proporciona algumas das propriedades mais importantes da
transição de estado físico de líquidos para gases. O valor de $L$ pode ser
extraido da linha de coexistência utilizando a equação de Clausius-Clapeyron
para transições de primeira ordem em fluidos

$$\left(\frac{dp}{dT}\right)_\text{coex}=\frac{L}{T(v_g-v_l)}.$$

Como tanto o volume quanto a entropia são diferentes nos lados da fronteira,
sendo primeiras derivadas da energia livre, essa transição é dita de primeira
ordem. Próximo ao ponto crítico, a linha de coexistência é
$$p_r-1=4(T_r-1),$$
com
$$v^{(i)}=1-2\sqrt{1-T_r},\quad v^{(f)}=1+2\sqrt{1-T_r},$$
e portanto o calor latente é
$$L=16p_cv_c\sqrt{1-T/T_c}.$$

\subsection{Problemas}

\begin{enumerate}
    \item
        Com um filamento elástico foram realizados experimentos e mediu-se que,
        para certo intervalo de temperatura $T$, a tração $f$ depende do
        comprimento $X$ como
        $$f=aX-bT+cTX.$$
        onde $a,b$ e $c$ foram medidas experimentalmente. Além disso,
        determinou-se que a capacidade térmica do elástico a comprimento
        constante é proporcional à temperatura $C_X=A(X)T.$
        \begin{enumerate}
            \item
                Calcule a variação de entropia por comprimento à temperatura
                constante $\pd{S}{X}{T}$ em termos de $a,b,c$.
                \answer{$\pd{S}{X}{T}=b-cX$}
            \item
                Mostre que $A$ na verdade é independente de $X$.
            \item
                Dê uma espressão para a entropia $S(T,X)$ do filamento como
                função da temperatura e do comprimento, dado que $S(0,0)=S_0$,
                em termos de $a,b,c$ e $A$.
                \answer{$S=S_0+AT+bX-\frac{1}{2}cX^2$}
            \item 
                Calcule a capacidade térmica à tração constante $C_f$ em termos
                de $a,b,c$ e $A$.
                \answer{$C_f=AT+\frac{aX}{a+cT}$}
        \end{enumerate}
    \item
        Mostre que estabilidade garante que as seguintes grandezas são positivas
        para um fluido simples
        \begin{enumerate}
            \item $C_V=T\pd{S}{T}{V}$
            \item $C_p=T\pd{S}{T}{p}$
            \item $\kappa_S=-\frac{1}{V}\pd{V}{p}{S}$
            \item $\kappa_T=-\frac{1}{V}\pd{V}{p}{T}$
        \end{enumerate}
    \item
        Mostre que o gás ideal é estável.
    \item
        Uma certa quantidade de água é refrigerada de $25\,\mathrm{^\circ C}$
        até $0\,\mathrm{^\circ C}$ e congelada. Neste processo todo o calor
        do refrigerador utilizado é depositado em uma mesma quantidade de água
        na máxima eficiência teórica. Essa segunda vai de $25\,\mathrm{^\circ C}
        $ até $100\,\mathrm{^\circ C}$ e uma fração é vaporizada. Dado que para
        a água $L_f=80\,\mathrm{cal/g}$ e $L_v=540\,\mathrm{cal/g}$, encontre a
        fração de água que foi vaporizada.
        \answer{
            $\eta=\frac{T_b}{L_v}\left(c_w\log\frac{T_0^2}{T_bT_f}+
            \frac{L_f}{T_f}\right)=14\%$
        }
    \item
        Supondo que o calor latente de vaporização da água é aproximadamente
        independente da temperatura $L=2300\,\mathrm{kJ/kg}$, e desprezando o
        volume da água líquida, encontre a temperatura de ebulição da água no
        topo do monte Everest, onde a pressão atmosférica é $\eta=33\%$ do que é
        no nível do mar.
        \answer{$\theta=70\,\mathrm{^\circ C}$}
    \item
        (200 PPP) Um bloco de gelo de $0{,}6\,\mathrm{kg}$ a $-10\,\mathrm{^
        \circ C}$ é colocado no interior de um recipiente evacuado fechado de
        $1\,\mathrm{m^3}$ que está à mesma temperatura. Aquecemos então o
        recipiente até $100\,\mathrm{^\circ C}$. Desprezando a capacidade
        térmica do recipiente, encontre o calor fornecido ao mesmo. É dado que
        $c_i=0{,}5\,\mathrm{cal/g\,^\circ C}$.
        \answer{
            $Q=m(c_i(T_f-T_0)+L_f+c_w(T_b-\theta_f)+L_v- \frac{RT_b}{\mu})=
            4100\,\mathrm{kcal}$
        }
    \item 
        (200 MPPP) Ar é uma mistura de oxigênio e nitrogênio. O ponto de
        ebulição do nitrogênio à pressão atmosférica é $77{,}4\,\mathrm K$. 
        Quando o ar é comprimido isotermicamente a $77{,}4\,\mathrm K$ até 
        $112\%$ da pressão atmosférica, observamos o princípio da liquefação do
        oxigênio. Sabendo que o ar é aproximadamente $20\%$ oxigênio, encontre a
        proporção de oxigênio numa mistura que, quando comprimida a 
        $77{,}4\,\mathrm K$, observamos a liquefação de ambos os gases ao mesmo
        tempo.
        \answer{$x'=\frac{1}{1+\frac{1}{\eta x}}=18\%$}

    \item
        Uma quantidade $N$ de quartzo líquido, se resfriado vagarosamente,
        cristaliza a uma temperatura $T_0$ liberando calor latente $NL$. Se
        resfriarmos rapidamente, o líquido sofre superresfriamento e torna-se
        vidro, sem liberar calor.
        \begin{enumerate}
            \item
                Nas duas fases podemos considerar o quartzo incompressível, sem
                apresentar qualquer forma de trabalho. Escreva o potencial
                químico $\mu$ em termos da energia interna $U$, entropia $S$,
                temperatura $T$ e quantidade $N$.
                \answer{$\mu=\frac{U-TS}{N}$}
            \item 
                A capacidade térmica da fase cristalina é $C_X=\alpha NT^3$,
                enquanto a da fase vitrificada é $C_G=\beta NT$. Supondo que a
                entropia das duas fases é a mesma à $T=0$, escolhida como nula,
                calcule a entropia das duas fases como função da temperatura.
                \answer{$S_X=\frac{1}{3}\alpha NT^3,S_G=\beta NT$}
            \item
                Faça o mesmo para a energia interna, supondo que é a mesma para
                ambas as fases em $T=0$ e escolhida como nula.
                \answer{$U_X=\frac{1}{4}\alpha NT^4,U_G=\frac{1}{2}\beta NT^2$}
            \item
                Encontre $T_0$.
                \answer{$T_0=\sqrt\frac{6\beta}{\alpha}$}
            \item
                Encontre $L$.
                \answer{$L=-\frac{6\beta^2}{\alpha}$}
            \item
                O resultado anterior pode estar correto? Se não, qual suposição
                feita no problema é mais provavelmente problemática?
                \answer{o calor latente obtido é negativo. não podemos supor que
                a entropia das duas fases é a mesma em $T=0$}
        \end{enumerate}
    \item
        (200 MPPP) Encontre o calor liberado por unidade de massa no
        congelamento de água superresfriada à temperatura $T_0=-10\,\mathrm
        {^\circ C}$.
        \answer{$L=T_0\left(\frac{L_f}{T_f}-(c_w-c_i)\log\frac{T_f}{T_0}\right)
        =70\,\mathrm{cal/g}$}
\end{enumerate}


\appendix

\section{Cálculo multivariável}

\epigraph{\justifying A Filosofia está escrita neste grande livro, o Universo,
que está permanentemente aberto e ao alcance do nosso olhar. Mas o livro não
pode ser compreendido sem antes aprendermos a linguagem e os caracteres em que
está escrito. A linguagem é a Matemática, e os caracteres são triângulos, 
círculos e outras figuras geométricas, sem as quais é humanamente impossível
compreender uma única palavra. Sem estes, ficamos perambulando dentro de um
escuro labirinto.} {\emph{Galileu Galilei}, Il Saggiatore (1623)}

\noindent As técnicas do estudo do comportamento local de funções podem ser
facilmente extendidas para funções cujo domínio, ao invés de intervalos em
$\mathbb R$, são bolas no $\mathbb R^n$. Muitos resultados tornam-se mais claros
nessa linguagem e portanto devemos dar uma boa atenção para ela. Nossos objetos
de estudo serão funções $f:\mathbb R^n\rightarrow\mathbb R$.

\subsection{A derivada direcional}

Começamos com a seguinte questão: como definir a noção de derivada, ou taxa de
variação, para uma função que depende de mais de um parâmetro? Lembrando que
quando haviamos apenas uma variável, a derivada de uma função $g:\mathbb R
\rightarrow\mathbb R$ era definida como
$$g'(t)=\lim\limits_{h\rightarrow0}\frac{g(t+h)-g(t)}{h}.$$
Para conseguimos utilizar essa mesma ideia em várias variáveis, considere alguma
trajetória no espaço $\mathbb R^n$,
$$\gamma(t)=(\gamma^1(t),\gamma^2(t),\dots,\gamma^n(t)).$$
A composição $f\circ\gamma$, portanto, é uma função em que podemos definir a
derivada tradicional:
$$(f\circ\gamma)'(t)=
\lim\limits_{h\rightarrow0}\frac{f(\gamma(t+h))-f(\gamma(t))}{h},$$
chamada de \emph{derivada direcional} ou \emph{derivada total} de $f$ em $\gamma
(t)$ ao longo de $\gamma'(t)$. No entanto, gostariamos de definir uma noção de
derivada independente da trajetória utilizada, e a maneira que escrevemos acima
parece sugerir que não há uma forma trivial de separar qual parte da derivada é
devido à função $f$, e qual é devido a nossa escolha de trajetória $\gamma$.
Somando vários zeros na expressão no interior do limite, no entanto:
\begin{align*}
    \frac{f(\gamma(t+h))-f(\gamma(t))}{h}
    &=\frac{f(\gamma^1(t+h),\gamma^2(t+h),\dots,\gamma^n(t+h))
    -f(\gamma^1(t),\gamma^2(t),\dots,\gamma^n(t))}{h}\\
    &=\frac{f(\gamma^1(t+h),\gamma^2(t+h),\dots,\gamma^n(t+h))
    -f(\gamma^1(t),\gamma^2(t+h),\dots,\gamma^n(t+h))}{h}\\
    &+\frac{f(\gamma^1(t),\gamma^2(t+h),\dots,\gamma^n(t+h))
    -f(\gamma^1(t),\gamma^2(t),\dots,\gamma^n(t+h))}{h}\\
    &+\cdots\\
    &+\frac{f(\gamma^1(t),\gamma^2(t),\dots,\gamma^n(t+h))
    -f(\gamma^1(t),\gamma^2(t),\dots,\gamma^n(t))}{h}
\end{align*}
e podemos usar uma lógica análoga à regra da cadeia para escrever o limite de
cada uma destas frações, afirmando que
$$\lim\limits_{h\rightarrow0}\frac{f(\dots,\gamma^i(t+h),\dots)
-f(\dots,\gamma^i(t),\dots)}{h}=f_{,i}(\gamma(t))\gamma^i{'}(t)$$
onde
$$f_{,i}(x)\equiv\lim\limits_{h\rightarrow0}
\frac{f(\dots,x^i+h,\dots)-f(\dots,x^i,\dots)}{h}$$
é chamada a \emph{derivada parcial} de $f$ na direção $x^i$. Essa é calculada
assim como uma derivada usual, somente na variável de interesse enquanto as
outras são consideradas coeficientes constantes. Outras notações comuns para
essa grandeza são
$$\partial_i f\quad\text{ou}\quad\pd{f}{x^i}{}.$$
Nossa expressão para a derivada total então fica
$$(f\circ\gamma)'(t)=f_{,1}(\gamma(t))\gamma^{1}{'}(t)
+\cdots+f_{,n}(\gamma(t))\gamma^{n}{'}(t).$$
Isso nos mostra que a derivada direcional depende apenas do valor e da primeira
derivada da trajetória. ou deixando a trajetória em questão implícita,
escrevemos uma variação genérica do valor de $f$ em termos das variações dos
valores das coordenadas como
$$df=f_{,1}dx^1+\cdots+f_{,n}dx^n.$$
Podemos definir também derivadas de maior ordem encadeando esse procedimento
recursivamente
$$f_{,i_1i_2\dots i_N}(x)=(f_{,i_2\dots i_N})_{,i_1}(x),$$
também com notações alternativas
$$\partial_{i_1\dots i_N} f\quad\text{ou}
\quad\frac{\partial^N f}{\partial x^{i_1}\cdots\partial x^{i_N}}.$$
Quando essas derivadas são contínuas, elas não dependem da ordem em que as
tomamos, valendo que
$$f_{,i_1\dots i_Nj_1\dots j_M}(x)=f_{,j_1\dots j_Mi_1\dots i_N}(x).$$

\subsection{Funções Implícitas}

Muitas vezes nos encontramos em situações onde ao invés de estar interessados em
funções, queremos entender o comportamento de variáveis relacionadas por
equações. Considere o caso em que temos $n+m$ variáveis que vamos denotar por
$x^1,\dots,x^n,y^1,\dots,y^m$, sujeitas a $m$ equações
\begin{align*}
    f_1(x^1,\dots,x^n,y^1,\dots,y^m)&=0,\\
    f_2(x^1,\dots,x^n,y^1,\dots,y^m)&=0,\\
    \cdots&\\
    f_m(x^1,\dots,x^n,y^1,\dots,y^m)&=0.
\end{align*}

Definimos neste caso o \emph{Jacobiano} das funções $f_1,\dots,f_m$ com relação
as variáveis $y^1,\dots,y^m$ como a matriz $J(x^1,\dots,x^n,y^1,\dots,y^m)$
cujas entradas são dadas por 
$$J=\left(f_{i,y^j}\right).$$
Assim, vale o seguinte 
\begin{theorem}[Função implícita]
    Dado $(x_0,y_0)=(x^1_0,\dots,x^n_0,y^1_0,\dots,y^m_0)$ tal que $f_1(x_0,y_0)
    =\cdots=f_m(x_0,y_0)=0$, se todas as componentes de $J$ são contínuas numa
    vizinhança de $(x_0,y_0)$ e $\det J(x_0,y_0)\neq0$, então existem funções
    únicas $\varphi^1(x),\dots,\varphi^m(x)$ tal que $\varphi^1(x_0)=y^1_0,
    \dots, \varphi^m(x_0)=y^m_0$ e $f_1(x,\varphi(x))=\cdots=f_m(x,\varphi(x))
    =0$ numa vizinhança de $x_0$.
\end{theorem}

Note que a separação entre as variáveis independentes $x$ e as dependentes $y$ é
arbitrária, e separações diferentes levam a construções de funções diferentes, e
que somente algumas podem satisfazer as hipóteses do teorema ao que se refere o
Jacobiano enquanto outras não.

Como rascunho de demonstração, considere que as variações dos $y^k$ devem ser
tais que, junto das variações dos $x^j$, cancelam qualquer tipo de variação nos
$f_i$:
\begin{align*}
    df_1&=\quad f_{1,x^1}dx^1+\cdots+
    f_{1,x^n}dx^n+f_{1,y^1}dy^1+\cdots+f_{1,y^m}dy^m=0\\
    df_2&=\quad f_{2,x^1}dx^1+\cdots+
    f_{2,x^n}dx^n+f_{2,y^1}dy^1+\cdots+f_{2,y^m}dy^m=0\\
    &\quad\cdots\\
    df_m&=\quad f_{m,x^1}dx^1+\cdots+
    f_{m,x^n}dx^n+f_{m,y^1}dy^1+\cdots+f_{m,y^m}dy^m=0
\end{align*}
Temos portanto o seguinte sistema linear de $m$ equações:
\begin{align*}
    J_{11}dy^1+\cdots+J_{1m}dy^m&=-f_{1,x^1}dx^1-\cdots-f_{1,x^n}dx^n\\
    J_{21}dy^1+\cdots+J_{2m}dy^m&=-f_{2,x^1}dx^1-\cdots-f_{2,x^n}dx^n\\
    \cdots&\\
    J_{m1}dy^1+\cdots+J_{mm}dy^m&=-f_{m,x^1}dx^1-\cdots-f_{m,x^n}dx^n
\end{align*}
que, dado a condição $\det J\neq0$ tem solução dada pela matriz inversa:
\begin{align*}
    dy^1&=-(J^{-1})^{11}\left(f_{1,x^1}dx^1+\cdots+f_{1,x^n}dx^n\right)-
    \cdots-(J^{-1})^{1m}\left(f_{m,x^1}dx^1+\cdots+f_{m,x^n}dx^n\right)\\
    dy^2&=-(J^{-1})^{21}\left(f_{1,x^1}dx^1+\cdots+f_{1,x^n}dx^n\right)-
    \cdots-(J^{-1})^{2m}\left(f_{m,x^1}dx^1+\cdots+f_{m,x^n}dx^n\right)\\
    \cdots&\\
    dy^m&=-(J^{-1})^{m1}\left(f_{1,x^1}dx^1+\cdots+f_{1,x^n}dx^n\right)-
    \cdots-(J^{-1})^{mm}\left(f_{m,x^1}dx^1+\cdots+f_{m,x^n}dx^n\right)\\
\end{align*}
Podemos então coletar os termos de $dx^j$ em cada uma destas equações e escrever
um sistema de $mn$ equações diferenciais
$$dy^i=\varphi^i_{,1}dx^1+\cdots+\varphi^i_{,n}dx^n \Rightarrow
\varphi^i_{,j}=-(J^{-1})^{i1}f_{1,x^j}-\cdots-(J^{-1})^{im}f_{m,x^j}.$$
Esse sistema de equações possui solução única dada a condição inicial $\varphi
(x_0)=y_0$, e vale na vizinhança de $x_0$ onde $\det J\neq0$.

Com esse maquinário podemos então definir as seguintes grandezas para todo $y^i$
e todo $x^j$:
$$\pd{y^i}{x^j}{x^1,\dots,x^{j-1},x^{j+1},\dots,x^n}\equiv\phi^i_{,j},$$
que é lido como a \emph{derivada parcial de $y^i$ com relação a $x^j$ a $x^1,
\dots,x^{j-1},x^{j+1} ,\dots,x^n$ constantes}. As variáveis constantes precisam
ser explicitadas para determinar qual foi a construção realizada com o teorema
da função implícita, mas pode ser omitida em casos em que a decomposição
utilizada fica clara do contexto. No caso $n=1$ denotamos essa grandeza
simplesmente com a notação de Leibniz usual, $\frac{dy^i}{dx}$. Isso nos permite
escrever a variação de cada $y^i$ como
$$dy^i=\pd{y^i}{x^1}{x^2,\dots,x^n}dx^1+\pd{y^i}{x^2}{x^1,x^3,\dots,x^n}dx^2
+\cdots+\pd{y^i}{x^n}{x^1,\dots,x^{n-1}}dx^n.$$
Derivadas de maior ordem podem também ser construidas com seu encadeamento,
desde que a decomposição utilizada no teorema da função implícita seja a mesma
para todas elas, e tem a mesma propriedade de simetria de permutação da derivada
parcial usual. No caso em que as variáveis dependentes já são funções explícitas
das variáveis dependentes, a solução do teorema da função implícita são as
funções em questão, e as derivadas parciais a variáveis constantes são apenas as
derivadas parciais das funções originais.

Por fim, uma trajetória $\gamma(t)$ nas variáveis independendes extende-se
naturalmente para as variáveis dependentes atrvés da função $\varphi$ como 
$(\gamma(t), \varphi\circ\gamma(t))$.

\subsection{Transformação de Legendre}

Suponha que estamos estudando a variável $y$ na relação $f(q,x,y)=0$ e estamos
particularmente interessados no valor da derivada 
$$p=\pd{y}{q}{x}$$
em vez do valor de $q$. Inspirando-se na seguinte manipulação:
$$dy=pdq+\pd{y}{x}{q}dx\Rightarrow d(y-pq)=-qdp+\pd{y}{x}{q}dx,$$
que sugere que a variável $z=y-pq$ é uma função implícita de $p$ sem ser função
implícita de $q$, vamos utilizar o teorema da função implícita no sistema de
equações
\begin{align*}
    f(q,x,z+pq)&=0\\
    f_{,y}(q,x,z+pq)p+f_{,q}(q,x,z+pq)&=0
\end{align*}
deixando $z$ e $q$ como função de $p$ e $x$. A conclusão que chegaremos é de que
$$dz=-qdp+\pd{y}{x}{q}dx,$$
ou seja que de fato
$$\pd{z}{p}{x}=-q,\quad\pd{z}{x}{p}=\pd{y}{x}{q}$$
e a condição no Jacobiano para que isso seja possível além da que já é tomada
como pressuposto quando tornamos $y$ função implícita torna-se simplesmente
$$\pd{p}{q}{x}\neq0.$$

A variável $z$ recebe o nome de \emph{transformação de Legendre} de $y$ em
relação a $q$. A construção acima pode ser facilmente extendida para uma relação
de várias variáveis $f(q^1,\dots,q^m,x^1,\dots,x^n,y)=0$, com
$$z=y-p_1q^1-\cdots-p_mq^m,$$
e para todos $i$ vale que
$$\pd{z}{p_i}{p_1,\dots,p_{i-1},p_{i+1},\dots,p_m,x^1,\dots,x^n}=-q^i,$$
$$\pd{z}{x^i}{p_1,\dots,p_m,x^1,\dots,x^{i-1},x^{i+1},\dots,x^n}=
\pd{y}{x^i}{q^1,\dots,q^m,x^1,\dots,x^{i-1},x^{i+1},\dots,x^n},$$
sob a condição no Jacobiano de $p$ em relação a $q$
$$\det\left(\pd{p_i}{q^j}{q^1,\dots,q^{j-1},q^{j+1},
\dots,q^m,x^1,\dots,x^n}\right)\neq0$$
A construção também claramente vale quando $n=0$ e todas as variáveis
independentes de uma relação sofrem a transformação ao mesmo tempo. Além disso,
tudo aqui vale trivialmente quando a variável dependente em questão é uma função
explícita das variáveis dependentes.

\subsection{Problemas}

\begin{enumerate}
    \item
        Considere a função $f(x,y) = 3x^2y-2x$ e a trajetória $\gamma(t)=
        (\cos t,\sin t)$.
        \begin{enumerate}
            \item
                Usando a definição de derivada direcional, encontre a derivada
                de $f$ em $\gamma(0)$ ao longo de $\gamma'(0)$
                \answer{$(f\circ\gamma)'(0)=3$}
            \item
                Utilize a decomposição em derivadas parciais e a velocidade da
                trajetória para obter o mesmo resultado
                \answer{$f_{,x}=6xy-2,f_{,y}=3x^2,\gamma'(t)=(-\sin t,\cos t)$}
        \end{enumerate}

        \item Encontre todas as derivadas parciais das seguintes funções:
        \begin{enumerate}
            \item 
                $f(x,y,z)=4x^3y^2-e^zy^4+\frac{z^3}{x^2}_4y-x^{16}$
                \answer{
                    $f_{,x}=12x^2y^2-\frac{2z^3}{x^3}-16x^{15},
                    f_{,y}=8x^3y-4e^zy^3+4,f_{,z}=-e^zy^4+\frac{3z^2}{x^2}$
                }
            \item 
                $g(s,t,u)=t^2\log(s+2t)-\log(3u)(s^3+t^2-4u)$
                \answer{
                    $g_{,s}=\frac{t^2}{s+2t}-3s^2\log(3u),
                    g_{,t}=2t\log(s+2t)+\frac{2t^2}{s+2t}-2t\log(3u),
                    g_{,u}=4\log(3u)-\frac{s^3+t^2-4u}{u}$
                }
        \end{enumerate}

    \item
        Dado o sistema de equações
        \begin{align*}
            x^2y+3y+z^3-z&=8\\
            2x+2y+\cos xz&=7
        \end{align*}
        que possui solução $(x,y,z)=(1,2,0)$, encontre um par de variáveis que
        podem ser escritas como função da outra em torno deste ponto e escreva
        as equações diferenciais associadas. Calcule o valor das derivadas no
        ponto dado.
        \answer{
            $\frac{dy}{dx}=-\frac{2x^2y\sin xz+(2-z\sin xz)(3z^2-1)}
            {(x^2+3)x\sin xz+2(3z^2-1)}=-1,
            \frac{dz}{dx}=-\frac{4xy-(x^2+3)(2-z\sin xz)}
            {(x^2+3)x\sin xz+2(3z^2-1)}=0$
        }

    \item
        A esfera unitária no $\mathbb R^3$ é definida pela equação $x^2+y^2+z^2
        =1$.
        \begin{enumerate}
            \item
                Encontre $z$ como função de $x,y$ de forma que a solução inclua
                o ponto $(0,0,1)$.
                \answer{$z=\sqrt{1-x^2-y^2}$}
            \item
                Utilize o teorema da função implícita em torno de $(0,0,1)$
                utilizando $x,y$ como variáveis independentes e $z$ como
                dependente para escrever um par de equações diferenciais. Mostre
                que sua resposta do item anterior satisfaz as equações.
                \answer{$\pd{z}{x}{y}=-\frac{x}{z},\pd{z}{y}{x}=-\frac{y}{z}$}
        \end{enumerate}

    \item
        Mostre que para uma relação $f(x,y,z)=0$ vale que
        \begin{align*}
            \pd{x}{y}{z}\pd{y}{x}{z}&=1\\
            \pd{x}{y}{z}\pd{y}{z}{x}\pd{z}{x}{y}&=-1
        \end{align*}

    \item
        Mostre que para um par de relações $f(x,y,z,w)=g(x,y,z,w)=0$ vale que
        \begin{align*}
            \pd{w}{z}{x}&=\pd{w}{y}{x}\pd{y}{z}{x}\\
            \pd{w}{x}{z}&=\pd{w}{x}{y}+\pd{w}{y}{x}\pd{y}{x}{z}
        \end{align*}

    \item
        Encontre a transformada de Legendre das seguintes funções
        \begin{enumerate}
            \item
                $f(t)=e^t$
                \answer{$g(s)=s(\log s-1)$}
            \item
                $x(q)=q^2$
                \answer{$y(p)=\frac{1}{4}p^2$}
        \end{enumerate}

    \item
        Suponha que uma variável $a$ dependa de $q$ e $x$ com o diferencial
        $$da=pdq+ydx.$$
        Mostre que
        $$\pd{a}{x}{p}=y-p\pd{y}{p}{x}$$

    \item
        * Ao final do século XVIII foi desenvolvida a mecânica analítica por
        Giuseppe Lagrangia. A abordagem consiste em resumir todos os efeitos da
        mecânica clássica através de uma função $L(q,v,t)$ que caracteriza o
        sistema em questão, dependendo da posição $q$, velocidade $v$ e do tempo
        $t$. As equações de movimento para a trajetória $\gamma_L(t)=
        (\gamma_L^q(t),\gamma_L^v(t),t)$ do sistema então são geradas pelo par
        de equações diferenciais
        \begin{align*}
            \gamma_L^q{'}(t)&=\gamma_L^v(t)\\
            \left(L_{,v}\circ\gamma_L\right)'(t)
            &=\left(L_{,q}\circ\gamma_L\right)(t)
        \end{align*}
        e condição inicial $\gamma_L(0)=(q_0,v_0,0)$. Começando com um exemplo,
        \begin{enumerate}
            \item
                Mostre que se $L(q,v,t)=\frac{1}{2}mv^2-\frac{1}{2}kq^2$
                obtemos a equação de movimento para a distensão $q$ de uma massa
                $m$ presa a uma mola de constante elástica $k$
        \end{enumerate}
        Porém o verdadeiro poder da abordagem que estamos considerando é a
        possibilidade de escrever equações de movimento utilizando sistemas de
        coordenadas alternativos quando os eixos cartesianos geram equações
        inelegantes. Suponha que gostariamos de descrever um sistema utilizando
        uma coordenada $Q$ que está relacionada à coordenada anterior por uma
        relação $f(q,Q,t)=0$.
        \begin{enumerate}
            \setcounter{enumii}{1}
            \item
                Encontre uma relação $g(q,v,Q,V,t)=0$ para uma nova variável $V$
                de forma que, utilizando o teorema da função implícita para
                construir $(q,v)=\varphi(Q,V,t)$, para qualquer trajetória 
                $\Gamma_L(t)=(\Gamma_L^Q(t),\Gamma_L^V(t),t)$
                $$\Gamma_L^Q{'}(t)=\Gamma_L^V(t)\Rightarrow(\varphi^q\circ
                \Gamma_L){'}(t)=(\varphi^v\circ\Gamma_L)(t)$$
                \answer{
                    $g(q,v,Q,V,t)=f_{,q}(q,Q,t)v+f_{,Q}(q,Q,t)V+f_{,t}(q,Q,t)$
                }
            \item
                Definindo
                $$\hat L(Q,V,t)=L(\varphi(Q,V,t),t)$$
                Mostre que se $\Gamma_L(t)$ for tal que 
                \begin{align*}
                    \Gamma_L^Q{'}(t)&=\Gamma_L^V(t)\\
                    \left(\hat L_{,V}\circ\Gamma_L\right)'(t)
                    &=\left(\hat L_{,Q}\circ\Gamma_L\right)(t)
                \end{align*}
                sujeito à condição inicial $(\varphi\circ\Gamma_L)(0)=(q_0,
                v_0)$, então $\gamma_L(t)=(\varphi\circ\Gamma_L,t)$.
        \end{enumerate}

        Outro ponto de vista da mecânica analítica que tem um sabor parecido com
        a de Lagrangia é o de William Hamilton. Aqui vamos caracterizar o
        sistema por uma função $H(q,p,t)$ da posição $q$, momento $p$ e tempo
        $t$. As equações de movimento para a trajetória $\gamma_H(t)=
        (\gamma_H^q(t),\gamma_H^p(t),t)$ agora tornam-se
        \begin{align*}
            \gamma_H^q{'}(t)&=(H_{,p}\circ\gamma_H)(t)\\
            \gamma_H^p{'}(t)&=-(H_{,q}\circ\gamma_H)(t)
        \end{align*}
        e condição inicial $\gamma_H(0)=(q_0,p_0,0)$.
        \begin{enumerate}
            \setcounter{enumii}{3}
            \item
                Mostre que se $H(q,p,t)=\frac{p^2}{2m}+\frac{1}{2}kq^2$ obtemos
                a equação de movimento para a distensão $q$ de uma massa $m$
                presa a uma mola de constante elástica $k$
        \end{enumerate}
        Note a grande simetria entre as duas equações quando comparada com as de
        Lagrangia. Isso nos permite fazer mudanças de coordenadas ainda mais
        esotéricas. Suponha que construimos um par de relações entre $q,p,Q,P,t$
        de forma que
        $$\pd{Q}{q}{p,t}\pd{P}{p}{q,t}-\pd{Q}{p}{q,t}\pd{P}{q}{p,t}=1.$$
        Neste caso a transformação de coordenadas é dita \emph{canônica}.
        \begin{enumerate}
            \setcounter{enumii}{4}
            \item
                Mostre que se uma transformação é canônica,
                $$\pd{q}{Q}{P,t}=\pd{P}{p}{q,t}$$
                e conclua que existe uma função $F(q,P,t)$ tal que $p=F_{,q}$ e
                $Q=F_{,P}$. Esta é chamada de \emph{função geratriz} da
                transformação canônica.
            \item
                Usando o teorema da função implícita para construir $(q,p)=\psi
                (Q,P,t)$ e definindo
                $$\hat H(Q,P,t)=H(\psi(Q,P,t),t)+F_{,t}(\psi^q(Q,P,t),P,t)$$
                Mostre que se $\Gamma_H(t)$ for tal que
                \begin{align*}
                    \Gamma_H^Q{'}(t)&=(\hat H_{,P}\circ\Gamma_H)(t)\\
                    \Gamma_H^P{'}(t)&=-(\hat H_{,Q}\circ\Gamma_H)(t)
                \end{align*}
                sujeito à condição inicial $(\psi\circ\Gamma_H)(0)=(q_0,p_0)$,
                então $\gamma_H(t)=(\psi\circ\Gamma_H)(t)$.
        \end{enumerate}
        Apesar do formalismo de Hamilton parecer mais poderoso que o de
        Lagrangia por podermos misturar posições e momentos, eles são em grande
        parte equivalentes, sendo relacionadas por uma transformação de Legendre
        entre as variáveis $v$ e $p$,
        $$H=-(L-pv).$$
        \begin{enumerate}
            \setcounter{enumii}{6}
            \item
                Caracterize a função geratriz da transformação de coordenadas
                estudada nos itens (b) e (c)
                \answer{$F(q,P,t)=P\varphi(q,t),\quad f(q,\varphi(q,t),t)=0$}
        \end{enumerate}
\end{enumerate}


%\section{Geometria Diferencial}

\subsection{Vetores no $\mathbb R^n$}

\subsection{Problemas}
\begin{enumerate}
    \item
        As coordenadas esféricas $r,\theta,\phi$ para o $\mathbb R^3$ são
        definidas de acordo com as seguintes equações a partir das coordenadas
        cartesianas $x,y,z$:
        \begin{align*}
            x&=r\sin\theta\cos\phi\\
            y&=r\sin\theta\sin\phi\\
            z&=r\cos\theta
        \end{align*}
        e podem ser visualizadas como na figura abaixo.
        \begin{figure}[H]
            \centering
            \begin{tikzpicture}
                \draw[thick,->]
                    (0,0) -- ({-2*tan(30)},-2) node[below]{$x$};
                \draw[thick,->]
                    (0,0) -- (3,0) node[right]{$y$};
                \draw[thick,->]
                    (0,0) -- (0,3) node[above]{$z$};
                \draw[ultra thick, ->]
                    (0,0) -- node[pos=0.5, below right] {$r$} (2,2);
                \draw[dashed]
                    (0,0) -- (2,{-2/tan(60)}) -- (2,2);
                \draw
                    (1.8,{-1.8/tan(60)}) -- (1.8,{-1.8/tan(60)+0.2})
                    -- (2,{-2/tan(60)+0.2});
                \fill
                    (1.9,{-1.9/tan(60)+0.1}) circle (0.02);
                \draw[dashed]
                    ({0.5*cos(240)},{0.5*sin(240)})
                    arc (240:330:0.5) node[pos=0.5,below] {$\phi$};
                \draw[dashed]
                    ({0.5*cos(45)},{0.5*sin(45)})
                    arc (45:90:0.5) node[pos=0.5,above] {$\theta$};
            \end{tikzpicture}
        \end{figure}

        Note que estas parametrizam o espaço usando $[0,\infty)\times[0,\pi]
        \times[0,2\pi)$. Encontre, como função de $r,\theta,\phi$
        \begin{enumerate}
            \item
                A base $e_r,e_\theta,e_\phi$ em termos da base $e_x,e_y,e_z$
                \answer{
                    $e_r=\sin\theta(\cos\phi e_x+\sin\phi e_y)+\cos\theta e_z$
                    \par$e_\theta=
                    r(\cos\theta(\cos\phi e_x+\sin\phi e_y)-\sin\theta e_z)$
                    \par$e_\phi=r\sin\theta(-\sin\phi e_x+\cos\phi e_y)$}
            \item
                A métrica para as coordenadas esféricas, supondo a métrica
                euclideana para as cartesianas.
                \answer{
                    $g=\left(\begin{matrix}
                        1&0&0\\
                        0&r^2&0\\
                        0&0&r^2\sin^2\theta
                    \end{matrix}\right)$
                }
            \item
                Vemos a partir do item anterior que as coordenadas esféricas
                formam um sistema ortogonal. Encontre os versores $\hat r,\hat
                \theta,\hat\phi$ em termos dos versores $\hat x, \hat y,\hat z$.
                \answer{
                    $\hat r=
                    \sin\theta(\cos\phi\hat x+\sin\phi\hat y)+\cos\theta\hat z$
                    \par$\hat\theta=
                    \cos\theta(\cos\phi\hat x+\sin\phi\hat y)-\sin\theta\hat z$
                    \par$\hat\phi=-\sin\phi\hat x+\cos\phi\hat y$
                }
        \end{enumerate}
\end{enumerate}


\section{Respostas dos problemas}

\renewcommand{\do}[1]{\noindent\par #1}
\dolistloop{\problemanswers}

\end{document}
