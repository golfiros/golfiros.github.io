\section{Transições de Fase}

\subsection{Potenciais Termodinâmicos}

Da maneira como foram formuladas as leis da termodinâmica, vemos que as
variáveis naturais para se trabalhar com o assunto são a energia interna, a
entropia, e as coordenadas de trabalho do sistema. No entanto, em muitos
contextos teóricos e experimentais nosso interesse é no que acontece quando
especificamos alguns parâmetros intensivos ao invés dos extensivos naturais.
Especificar a pressão de um gás ao invés de seu volume. A tensão num elástico ao
invés de sua distensão. A temperatura de um objeto ao invés de sua entropia. 

Fisicamente, há uma maneira muito simples de implementar este tipo de vínculo.
Suponha que a primeira lei da termodinâmica para um sistema físico pode ser
escrita
$$dU=p_1dQ_1+\cdots+p_ndQ_n+y_1dX_1+\cdots+y_ndX_n,$$
onde chamamos $p,Q$ variáveis que queremos controlar o lado intensivo, enquanto
$y,X$ as que queremos controlar o lado extensivo. Ao invés de tratarmos o
sistema como isolado, acomplamos o mesmo a algum tipo de reservatório que pode
absorver as variáveis extensivas conjugadas às intensivas que estamos tentando
fixar, sendo este grande o suficiente de forma que possamos aproximar sua
energia interna até primeira ordem:
$$U_\text{res}(Q_1,\dots,Q_m)
\approx U^{(0)}+p^{(0)}_1(Q_1-Q^{(0)}_1)+\cdots+p^{(0)}_m(Q_m-Q^{(0)}_m).$$

Realizamos o acoplamento olhando então para a energia total do conjunto,
$$U_\text{tot}=U(Q_1,\dots,Q_m,X_1,\dots,X_n)
+U_\text{res}(Q^{(0)}_1-Q_1,\dots,Q_m)$$
e deixamos o sistema encontrar seu equilíbrio isolado do resto do universo. Isso
é equivalente a encontrar os $Q$ que minimizam a expressão
$$V=U-p^{(0)}_1Q_1-\cdots-p^{(0)}_mQ_m,$$
levando-nos a concluir que as variáveis intensivas devem ter o mesmo valor que
têm no reservatório.
$$p_1=p^{(0)}_1,\dots,p_m=p^{(0)}_m.$$
Podemos resumir essa construção analisando a função $V(p_1,\dots,p_m,X_1,\dots,
X_n)$ definida como a transformação de Legendre de $U$ nas variáveis de
interesse, e recebe o nome de um \emph{potencial termodinâmico}. Como nestes
casos estamos normalmente nos preocupando apenas com a configuração das
variáveis $X$, a caracterização do equilíbrio como ponto que minimiza $V$
continua sendo válida. Note também que dada a extensividade, se fizermos a
transformação utilizando todas as variáveis extensivas, a função que sobra é
identicamente nula e perdemos qualquer informação sobre o que é sistema e o que
é reservatório. Precisamos pelo menos de uma grandeza extensiva de referência
nos nossos potenciais termodinâmicos. 

Alguns casos específicos dessa construção recebem nomes. Para qualquer sistema
podemos realizar a transformação $F=U-TS$, que recebe o nome de \emph{energia
livre de Helmholtz}. Quando lidamos com gases e outros fluidos que possuem
pressão e volume, definimos também a \emph{entalpia} $H=U+pV$ e a \emph{energia
livre de Gibbs} $G=U-TS+pV$. Ainda para um fluido, tomando a segunda derivada
dessas funções em diferentes ordens nos permite extrair as \emph{relações de
Maxwell}:
\begin{alignat*}{3}
    &&U_{,SV}&=\pd{T}{V}{S}&&=-\pd{p}{S}{V}\\
    &&H_{,Sp}&=\pd{T}{p}{S}&&=\pd{V}{S}{p}\\
    &&-F_{,TV}&=\pd{S}{V}{T}&&=\pd{P}{T}{V}\\
    &&-G_{,Tp}&=\pd{S}{p}{T}&&=-\pd{V}{T}{p}
\end{alignat*}

\subsection{Estabilidade}

Vamos considerar um sistema termodinâmico em que escrevemos sua entropia em
termos das variáveis extensivas $S(X)$. De maneira geral visualizamos essa
situação como as variáveis extensivas uniformemente distribuidas ao longo da
extensão do corpo, porém na realidade essas grandezes flutuam ao longo do tempo,
de forma que a princípio podemos observar uma parcela $\Delta$ em proporções
diferentes do resto do corpo $X-\Delta$:
\begin{figure}[H]
    \centering
    \begin{tikzpicture}
        \begin{scope}[xshift=-2.5cm]
            \draw[very thick,fill=lightgray]
                (0,0) -- (3,0) -- (3,3) -- (0,3) -- cycle;
            \node
                at (1.5,1.5) {$X$};
        \end{scope}
        \draw[ultra thick,->]
            (1,1.5) -- (2,1.5);
        \begin{scope}[xshift=2.5cm]
            \draw[very thick,fill=lightgray]
                (0,0) -- (3,0) -- (3,3) -- (0,3) -- cycle;
            \draw[thick,fill=gray,use Hobby shortcut,closed=true] 
                (0.7,1.5) .. (1.2,1.8) .. (1.1,2.0) .. (1.3,2.2) .. (0.7,2.3);
            \node
                at (0.85,2) {$\Delta$};
            \node
                at (2,0.8) {$X-\Delta$};
        \end{scope}
    \end{tikzpicture}
\end{figure}
Nessa situação a entropia varia de $S(X)$ para $S(\Delta)+S(X-\Delta)$, e
portanto será proibida pela Segunda Lei quando isso corresponder a uma
diminuição de entropia. Ou seja, concluimos que os pontos de equilíbrio
possíveis para nosso sistema são apenas aqueles $X$ onde
$$S(X)\geq S(X-\Delta)+S(\Delta),$$
para toda variação $\Delta$, e a entropia é uma função \emph{côncava} das
variáveis de estado extensivas. Do contrário, observariamos rapidamente
fenômenos de \emph{nucleação} como descritos acima. O caso de igualdade precisa
ser incluido devido a extensividade, e só é encontrado no caso em que $\Delta
\sim X$.

Supondo agora apenas que a temperatura termodinâmica do nosso sistema é
positiva, uma simples manipulação análoga à que foi feita na condição de
equiilíbrio nos leva também a concluir a \emph{convexidade} da energia interna
do sistema em termos das variáveis extensivas:
$$U(X)\leq U(X-\Delta)+U(\Delta).$$
Lembrando da expressão da primeira lei para processos contínuos, podemos também
escrever uma relação entre os diferenciais das variáveis de estado que precisa
ser satisfeita para todo processo possível em qualquer estado de equilíbrio
estável:
$$dy_1dX_1+\cdots+dy_ndX_n\geq0,$$
onde a igualdade segue da relação de Gibbs-Duhem em uma transformação de escala
dada por $dX\sim X$. Essa condição de estabilidade local pode ser satisfeita até
por estados que violam a estabilidade global, levando à \textit{metaestabilidade
} de alguns estados, que são resistentes a pequenas flutuações mas que estão
sujeitos a nucleação para choques suficientemente grandes.

Para outros potenciais termodinâmicos, novamente nos interessamos apenas na
variação das variáveis extensivas que sobraram, e é fácil de ver que a
convexidade nestas é preservada pela transformação de Legendre.

\subsection{Fases da Matéria}

Suponha que certo sistema possui entropia
$$S(X_1,\dots,X_n,N),$$
onde denotamos uma variável extensiva $N$ específica que vamos utilizar para
controlar o seu tamanho. Podemos definir então a \emph{entropia específica} 
$$s(x_1,\dots,x_n)=\frac{S(Nx_1,\dots,Nx_n,N)}{N},$$
que por extensividade está bem definida independente do valor de $N$ que
escolhermos, e as variáveis intensivas $x_1,\dots,x_n$ são chamadas \emph{
específicas}. Neste caso, ganhamos um novo sentido para a questão da
estabilidade. Um estado $(Nx,N)$ será estável se, e somente se para quaisquer
pares de pontos $x^{(0)},x^{(1)}$ tais que existe $0<t<1$ de forma que $x=(1-t)
x^{(0)}+ tx^{(1)}$, vale que
$$s(x)>(1-t)s(x^{(0)})+ts(x^{(1)}).$$
Caso contrário, o sistema se decompõe em duas \emph{fases} de estados $((1-t)
x^{(0)},(1-t)N)$ e $(tx^{(1)},tN)$. Geometricamente, vemos que um estado $x$ é
estável quando o gráfico de $s$ fica abaixo do plano tangente ao mesmo em $x$.
No gráfico abaixo vemos um exemplo de região instável no caso de uma única
variável $x$.
\begin{figure}[H]
    \centering
    \begin{tikzpicture}
        \draw[very thick,->] 
            (0,0) -- (8,0) node[right] {$x$};
        \draw[very thick,->]
            (0,0) -- (0,3) node[left] {$s$};
            \begin{luacode}
                function f(x, m)
                    t = math.pi * (x - 2.5) / (5.5 - 2.5)
                    return 
                        -1.2 * math.log(x) 
                        + 0.5 * math.sin(t)^2
                        + m * x
                end

                function d(m)
                    local x0 = minimize(f, {2.5, 3.5}, m)
                    local x1 = minimize(f, {4.5, 5.5}, m)
                    return f(x1, m) - f(x0, m)
                end

                m = solve(d, {0.0, 0.5})
                x0 = minimize(f, {2.5, 3.5}, m)
                x1 = minimize(f, {4.5, 5.5}, m)

                tex.print(
                    "\\draw[fill=lightgray,thick,dashed]",
                    "(", x0, ",0) -- (", x0,",", -f(x0, 0), ")", 
                    "-- (", x1, ",", -f(x1, 0), ") -- (", x1, ",0);"
                )

                function g(x) 
                    t = math.pi * (x - 2.5) / (5.5 - 2.5)
                    return -1.2 / x^2 - math.cos(2 * t) / (5.5 - 2.5)^2
                end

                y0 = solve(g, {2.5, 4.0})
                y1 = solve(g, {4.0, 5.5})

                tex.print(
                    "\\draw[ultra thick,samples=10]",
                    "plot[domain=1:2.5]",
                    "(\\x,{1.2*ln(\\x)})",
                    "plot[domain=2.5:", y0, "]",
                    "(\\x,{1.2*ln(\\x)",
                    "-0.5*pow(sin((\\x-2.5)*180/(5.5-2.5)),2)});"
                )
                tex.print(
                    "\\draw[ultra thick,dashed,domain=", 
                    y0, ":", y1, ",samples=10]",
                    "plot (\\x,{1.2*ln(\\x)",
                    "-0.5*pow(sin((\\x-2.5)*180/(5.5-2.5)),2)});"
                )
                tex.print(
                    "\\draw[ultra thick,samples=10]",
                    "plot[domain=", y1, ":5.5]",
                    "(\\x,{1.2*ln(\\x)",
                    "-0.5*pow(sin((\\x-2.5)*180/(5.5-2.5)),2)})",
                    "plot[domain=5.5:7,samples=5]",
                    "(\\x,{1.2*ln(\\x)});"
                )
            \end{luacode}
    \end{tikzpicture}
\end{figure}
A região metaestável foi diferenciada da totalmente instável pela linha
pontilhada. Qualquer estado preparado com $x$ na região instável gera uma
coexistência de fases na sua fronteira, e como a construção vem de retas
tangentes vemos que todas as fases da matéria que coexistem devem compartilhar o
valor de todos os coeficientes de trabalho - estão em equilíbrio termodinâmico
umas com as outras. Assim, quando analisamos transições de fase é natural
utilizar um potencial termodinâmico que deixa esses coeficientes intensivos em
primeiro plano. Escrevendo a Primeira Lei como
$$dU=y_1dX_1+\cdots y_ndX_n+\mu dN,$$
olhamos para o potencial $V=U-y_1X_1-\cdots y_nX_n$, genericamente chamado de
energia livre, que por extensividade é simplesmente dado por
$$V=\mu N.$$
Assim, identificamos pontos nessa fronteira encontrando processos contínuos $i
\rightarrow f$ tais que $y_1^{(i)}=y_1^{(f)},\dots,y_n^{(i)}=y_n^{(f)}$ e
$$\mu^{(i)}=\mu^{(f)}\Rightarrow\Delta V=-\int_i^fX_1dy_1+\cdots+X_ndy_n=0.$$
Apesar de que os valores dos coeficientes de trabalho devem ser o mesmo de ambos
os lados do processo, podemos ter mudanças nas propriedades das coordenadas de
trabalho. Mais precisamente, observamos possíveis diferenças nos valores ou nas
derivadas destes. A \emph{ordem} de uma transição de fase é o nome que damos
para a menor ordem de derivada da energia livre que é diferente antes e depois
da transição. Uma simples contagem de equações e variáveis livres também nos
mostra que o espaço dos coeficientes de trabalho onde $k$ fases podem coexistir
é $(n+1-k)$--dimensional.

\subsection{O Gás de van der Waals}

No caso de um fluido simples que possui energia interna, volume e número de
moléculas, o parâmetro $N$ utilizado normalmente é o número de moléculas e o
coeficiente $\mu$ recebe o nome de \emph{potencial químico}, e o potencial
termodinâmico de interesse é a energia livre de Gibbs. Considere portanto o gás
ideal
$$S=\frac{NR}{\gamma-1}\log\frac{UV^{\gamma-1}}{\Phi N^\gamma}.$$
Podemos incrementar o modelo para melhor representar gases reais através das
modificações
\begin{align*}
    U&\mapsto U+a\frac{N^2}{V}\\
    V&\mapsto V-Nb
\end{align*}
que correspondem respectivamente a uma força atrativa que tende a manter a
coesão do fluido, e a um limite intrínseco a quanto podemos comprimí-lo. 
As equações de estado correspondentes à lei dos gases ideais então imediatamente
tornam-se
$$U=\frac{NRT}{\gamma-1}-a\frac{N^2}{V},$$
$$\left(p+a\frac{N^2}{V^2}\right)\left(\frac{V}{N}-b\right)=RT.$$
Mas ao fazermos o gráfico de algumas das isotermas imediatamente nos deparamos
com um problema:
\begin{figure}[H]
    \centering
    \begin{tikzpicture}
        \draw[very thick,->]
            (0,0) -- (4,0) node[below] {$V$};
        \draw[very thick,->]
            (0,0) -- (0,3) node[left] {$p$};
        \begin{luacode}
            t = {0.86, 1.00, 1.1}
            f = {"black", "darkgray", "gray"}

            function p(v, t, c)
                return 1.5 * (8 * t / (3 * v - 1) - 3 / (v * v)) - c
            end
            
            for i = 1, 3 do
                tex.print(
                    "\\draw[ultra thick,samples=20,", f[i], "]",
                    "plot[domain=", solve(p, {0.4, 1}, t[i], 2.8), ":1]",
                    "(\\x,{1.5*(8*", t[i], "/(3*\\x-1)-3/(\\x*\\x))})",
                    "plot[domain=1:3.5]",
                    "(\\x,{1.5*(8*", t[i], "/(3*\\x-1)-3/(\\x*\\x))});"
                )
            end
        \end{luacode}
    \end{tikzpicture}
\end{figure}
\noindent Em uma delas temos uma região onde a pressão aumenta com o volume à
temperatura constante, uma clara violação das condições de estabilidade.

Ao invés de descartar o modelo, vamos tentar caracterizar a transição de fase
que está implícita nas regiões instáveis. Primeiro vemos que o comportamento só
aparece em isotermas de temperatura
$$T<T_c=\frac{8a}{27Rb},$$
chamada temperatura \emph{crítica} do modelo, que é a isoterma específica
apresentando um ponto de inflexão horizontal em
$$V=Nv_c=3Nb,\quad p=p_c=\frac{a}{27b^2}.$$
Podemos então colocar a equação de estado em termos do que chamamos de variáveis
\emph{reduzidas} para o gás de van der Waals,
$$T_r=T/T_c,\quad v_r=V/Nv_c,\quad p_r=p/p_c,$$
$$p_r=\frac{8T_r}{3v_r-1}-\frac{3}{v_r^2}.$$
A variação da energia livre de Gibbs ao longo de uma isoterma pode então ser
escrita
$$dG=Vdp\sim v_rdp_r,$$
e impondo que esta seja zero entre ambas as fronteiras de dada isoterma obtemos
a \emph{construção de Maxwell} para o gás de van der Waals
\begin{figure}[H]
    \centering
    \begin{tikzpicture}
        \draw[very thick,->]
            (0,0) -- (5,0) node[below] {$v_r$};
        \draw[very thick,->]
            (0,0) -- (0,3.5) node[left] {$p_r$};
        \begin{luacode}
            t = 0.86

            function p(v, k, p0)
                return k * (8 * t / (3 * v - 1) - 3 / (v * v)) - p0
            end

            vlmin = 0.5
            vlmax = minimize(p, {vlmin, 1}, 1, 0)

            vgmax = 5.0
            vgmin = minimize(p, {1, vgmax}, -1, 0)

            function g(v)
                return 
                    4 * t * (1 / (3 * v - 1) - math.log(3 * v - 1)) / 9
                    - 1 / v
            end

            function dg(p0)
                local vl = solve(p, {vlmin, vlmax}, 1, p0)
                local vg = solve(p, {vgmin, vgmax}, 1, p0)
                return g(vg) - g(vl)
            end

            p0 = solve(dg, {0.13, 0.63})
            vl = solve(p, {vlmin, vlmax}, 1, p0)
            vg = solve(p, {vgmin, vgmax}, 1, p0)
            vm = solve(p, {vlmax, vgmin}, 1, p0)

            tex.print(
                "\\draw (0.1,", 3 * p0, ") -- (-0.1,", 3 * p0, ")",
                "node[left] {$p^{(0)}$};"
            )
            tex.print(
                "\\draw (", vl, ",0.1) -- (", vl, ",-0.1)",
                "node[below] {$v^{(i)}$};"
            )
            tex.print(
                "\\draw (", vg, ",0.1) -- (", vg, ",-0.1)",
                "node[below] {$v^{(f)}$};"
            )

            tex.print(
                "\\draw[very thick,dashed,samples=10,fill=lightgray]",
                "plot[domain=", vl, ":", vm, "]",
                "(\\x,{3*(8*", t, "/(3*\\x-1)-3/(\\x*\\x))});"
            )

            tex.print(
                "\\draw[very thick,dashed,samples=10,fill=gray]",
                "plot[domain=", vm, ":", vg, "]",
                "(\\x,{3*(8*", t, "/(3*\\x-1)-3/(\\x*\\x))});"
            )

            tex.print(
                "\\draw[ultra thick,samples=10]",
                "plot[domain=", solve(p, {0.4, 1}, 1, 3.2 / 3), ":", vl, "]",
                "(\\x,{3*(8*", t, "/(3*\\x-1)-3/(\\x*\\x))})",
                "-- plot[domain=", vg, ":4.5]",
                "(\\x,{3*(8*", t, "/(3*\\x-1)-3/(\\x*\\x))});"
            )
        \end{luacode}
    \end{tikzpicture}
\end{figure}
\noindent que corresponde à igualdade das áreas sombreadas no gráfico. Vemos que
a transição se dá à pressão $p=p^{(0)}p_c$ entre uma fase densa incompressível 
de volume $Nv_l=Nv^{(i)}v_c$ até uma fase rarefeita compressível de volume $Nv_g
=Nv^{(f)}v_c$. Devido à expansão na transição, vemos a partir da conservação de
energia que o sistema precisa absorver calor $Q=NL>0$. A instabilidade no gas de
van der Waals nos proporciona algumas das propriedades mais importantes da
transição de estado físico de líquidos para gases. O valor de $L$ pode ser
extraido da linha de coexistência utilizando a equação de Clausius-Clapeyron
para transições de primeira ordem em fluidos

$$\left(\frac{dp}{dT}\right)_\text{coex}=\frac{L}{T(v_g-v_l)}.$$

Como tanto o volume quanto a entropia são diferentes nos lados da fronteira,
sendo primeiras derivadas da energia livre, essa transição é dita de primeira
ordem. Próximo ao ponto crítico, a linha de coexistência é
$$p_r-1=4(T_r-1),$$
com
$$v^{(i)}=1-2\sqrt{1-T_r},\quad v^{(f)}=1+2\sqrt{1-T_r},$$
e portanto o calor latente é
$$L=16p_cv_c\sqrt{1-T/T_c}.$$

\subsection{Problemas}

\begin{enumerate}
    \item
        Com um filamento elástico foram realizados experimentos e mediu-se que,
        para certo intervalo de temperatura $T$, a tração $f$ depende do
        comprimento $X$ como
        $$f=aX-bT+cTX.$$
        onde $a,b$ e $c$ foram medidas experimentalmente. Além disso,
        determinou-se que a capacidade térmica do elástico a comprimento
        constante é proporcional à temperatura $C_X=A(X)T.$
        \begin{enumerate}
            \item
                Calcule a variação de entropia por comprimento à temperatura
                constante $\pd{S}{X}{T}$ em termos de $a,b,c$.
                \answer{$\pd{S}{X}{T}=b-cX$}
            \item
                Mostre que $A$ na verdade é independente de $X$.
            \item
                Dê uma espressão para a entropia $S(T,X)$ do filamento como
                função da temperatura e do comprimento, dado que $S(0,0)=S_0$,
                em termos de $a,b,c$ e $A$.
                \answer{$S=S_0+AT+bX-\frac{1}{2}cX^2$}
            \item 
                Calcule a capacidade térmica à tração constante $C_f$ em termos
                de $a,b,c$ e $A$.
                \answer{$C_f=AT+\frac{aX}{a+cT}$}
        \end{enumerate}
    \item
        Mostre que estabilidade garante que as seguintes grandezas são positivas
        para um fluido simples
        \begin{enumerate}
            \item $C_V=T\pd{S}{T}{V}$
            \item $C_p=T\pd{S}{T}{p}$
            \item $\kappa_S=-\frac{1}{V}\pd{V}{p}{S}$
            \item $\kappa_T=-\frac{1}{V}\pd{V}{p}{T}$
        \end{enumerate}
    \item
        Mostre que o gás ideal é estável.
    \item
        Uma certa quantidade de água é refrigerada de $25\,\mathrm{^\circ C}$
        até $0\,\mathrm{^\circ C}$ e congelada. Neste processo todo o calor
        do refrigerador utilizado é depositado em uma mesma quantidade de água
        na máxima eficiência teórica. Essa segunda vai de $25\,\mathrm{^\circ C}
        $ até $100\,\mathrm{^\circ C}$ e uma fração é vaporizada. Dado que para
        a água $L_f=80\,\mathrm{cal/g}$ e $L_v=540\,\mathrm{cal/g}$, encontre a
        fração de água que foi vaporizada.
        \answer{
            $\eta=\frac{T_b}{L_v}\left(c_w\log\frac{T_0^2}{T_bT_f}+
            \frac{L_f}{T_f}\right)=14\%$
        }
    \item
        Supondo que o calor latente de vaporização da água é aproximadamente
        independente da temperatura $L=2300\,\mathrm{kJ/kg}$, e desprezando o
        volume da água líquida, encontre a temperatura de ebulição da água no
        topo do monte Everest, onde a pressão atmosférica é $\eta=33\%$ do que é
        no nível do mar.
        \answer{$\theta=70\,\mathrm{^\circ C}$}
    \item
        (200 PPP) Um bloco de gelo de $0{,}6\,\mathrm{kg}$ a $-10\,\mathrm{^
        \circ C}$ é colocado no interior de um recipiente evacuado fechado de
        $1\,\mathrm{m^3}$ que está à mesma temperatura. Aquecemos então o
        recipiente até $100\,\mathrm{^\circ C}$. Desprezando a capacidade
        térmica do recipiente, encontre o calor fornecido ao mesmo. É dado que
        $c_i=0{,}5\,\mathrm{cal/g\,^\circ C}$.
        \answer{
            $Q=m(c_i(T_f-T_0)+L_f+c_w(T_b-\theta_f)+L_v- \frac{RT_b}{\mu})=
            4100\,\mathrm{kcal}$
        }
    \item 
        (200 MPPP) Ar é uma mistura de oxigênio e nitrogênio. O ponto de
        ebulição do nitrogênio à pressão atmosférica é $77{,}4\,\mathrm K$. 
        Quando o ar é comprimido isotermicamente a $77{,}4\,\mathrm K$ até 
        $112\%$ da pressão atmosférica, observamos o princípio da liquefação do
        oxigênio. Sabendo que o ar é aproximadamente $20\%$ oxigênio, encontre a
        proporção de oxigênio numa mistura que, quando comprimida a 
        $77{,}4\,\mathrm K$, observamos a liquefação de ambos os gases ao mesmo
        tempo.
        \answer{$x'=\frac{1}{1+\frac{1}{\eta x}}=18\%$}

    \item
        Uma quantidade $N$ de quartzo líquido, se resfriado vagarosamente,
        cristaliza a uma temperatura $T_0$ liberando calor latente $NL$. Se
        resfriarmos rapidamente, o líquido sofre superresfriamento e torna-se
        vidro, sem liberar calor.
        \begin{enumerate}
            \item
                Nas duas fases podemos considerar o quartzo incompressível, sem
                apresentar qualquer forma de trabalho. Escreva o potencial
                químico $\mu$ em termos da energia interna $U$, entropia $S$,
                temperatura $T$ e quantidade $N$.
                \answer{$\mu=\frac{U-TS}{N}$}
            \item 
                A capacidade térmica da fase cristalina é $C_X=\alpha NT^3$,
                enquanto a da fase vitrificada é $C_G=\beta NT$. Supondo que a
                entropia das duas fases é a mesma à $T=0$, escolhida como nula,
                calcule a entropia das duas fases como função da temperatura.
                \answer{$S_X=\frac{1}{3}\alpha NT^3,S_G=\beta NT$}
            \item
                Faça o mesmo para a energia interna, supondo que é a mesma para
                ambas as fases em $T=0$ e escolhida como nula.
                \answer{$U_X=\frac{1}{4}\alpha NT^4,U_G=\frac{1}{2}\beta NT^2$}
            \item
                Encontre $T_0$.
                \answer{$T_0=\sqrt\frac{6\beta}{\alpha}$}
            \item
                Encontre $L$.
                \answer{$L=-\frac{6\beta^2}{\alpha}$}
            \item
                O resultado anterior pode estar correto? Se não, qual suposição
                feita no problema é mais provavelmente problemática?
                \answer{o calor latente obtido é negativo. não podemos supor que
                a entropia das duas fases é a mesma em $T=0$}
        \end{enumerate}
    \item
        (200 MPPP) Encontre o calor liberado por unidade de massa no
        congelamento de água superresfriada à temperatura $T_0=-10\,\mathrm
        {^\circ C}$.
        \answer{$L=T_0\left(\frac{L_f}{T_f}-(c_w-c_i)\log\frac{T_f}{T_0}\right)
        =70\,\mathrm{cal/g}$}
\end{enumerate}
