\section{Transições de Fase}

\subsection{Potenciais Termodinâmicos}

Da maneira como foram formuladas as leis da termodinâmica, vemos que as
variáveis naturais para se trabalhar com o assunto são a energia interna, a
entropia, e as coordenadas de trabalho do sistema. No entanto, em muitos
contextos teóricos e experimentais nosso interesse é no que acontece quando
especificamos alguns parâmetros intensivos ao invés dos extensivos naturais.
Especificar a pressão de um gás ao invés de seu volume. A tensão num elástico ao
invés de sua distensão. A temperatura de um objeto ao invés de sua entropia. 

Fisicamente, há uma maneira muito simples de implementar este tipo de vínculo.
Suponha que a primeira lei da termodinâmica para um sistema físico pode ser
escrita
$$dU=p_1dQ_1+\cdots+p_ndQ_n+y_1dX_1+\cdots+y_ndX_n,$$
onde chamamos $p,Q$ variáveis que queremos controlar o lado intensivo, enquanto
$y,X$ as que queremos controlar o lado extensivo. Ao invés de tratarmos o
sistema como isolado, acomplamos o mesmo a algum tipo de reservatório que pode
absorver as variáveis extensivas conjugadas às intensivas que estamos tentando
fixar, sendo este grande o suficiente de forma que possamos aproximar sua
energia interna até primeira ordem:
$$U_\text{res}(Q_1,\dots,Q_m)
\approx U^{(0)}+p^{(0)}_1(Q_1-Q^{(0)}_1)+\cdots+p^{(0)}_m(Q_m-Q^{(0)}_m).$$

Realizamos o acoplamento olhando então para a energia total do conjunto,
$$U_\text{tot}=U(Q_1,\dots,Q_m,X_1,\dots,X_n)
+U_\text{res}(Q^{(0)}_1-Q_1,\dots,Q_m)$$
e deixamos o sistema encontrar seu equilíbrio isolado do resto do universo. Isso
é equivalente a encontrar os $Q$ que minimizam a expressão
$$V=U-p^{(0)}_1Q_1+-\cdots-p^{(0)}_mQ_m,$$
levando-nos a concluir que as variáveis intensivas devem ter o mesmo valor que
têm no reservatório.
$$p_1=p^{(0)}_1,\dots,p_m=p^{(0)}_m.$$
Podemos resumir essa construção analisando a função $V(p_1,\dots,p_m,X_1,\dots,
X_n)$ definida como a transformação de Legendre de $U$ nas variáveis de
interesse, e recebe o nome de um \emph{potencial termodinâmico}. Como nestes
casos estamos normalmente nos preocupando apenas com a configuração das
variáveis $X$, a caracterização do equilíbrio como ponto que minimiza $V$
continua sendo válida. Note também que dada a extensividade, se fizermos a
transformação utilizando todas as variáveis extensivas, a função que sobra é
identicamente nula e perdemos qualquer informação sobre o que é sistema e o que
é reservatório. Precisamos pelo menos de uma grandeza extensiva de referência
nos nossos potenciais termodinâmicos. Alguns casos específicos dessa construção
recebem nomes. Para qualquer sistema podemos realizar a transformação $F=U-TS$,
que recebe o nome de \emph{energia livre de Helmholtz}. Quando lidamos com gases
e outros fluidos que possuem pressão e volume, definimos também a \emph{
entalpia} $H=U+pV$ e a \emph{energia livre de Gibbs} $G=U-TS+pV$. Ainda para um
fluido, tomando a segunda derivada dessas funções em diferentes ordens
nos permite extrair as \emph{relações de Maxwell}:
\begin{alignat*}{3}
    &&U_{,SV}&=\pd{T}{V}{S}&&=-\pd{p}{S}{V}\\
    &&H_{,Sp}&=\pd{T}{p}{S}&&=\pd{V}{S}{p}\\
    &&-F_{,TV}&=\pd{S}{V}{T}&&=\pd{P}{T}{V}\\
    &&-G_{,Tp}&=\pd{S}{p}{T}&&=-\pd{V}{T}{p}
\end{alignat*}

\subsection{Estabilidade}

Vamos considerar um sistema termodinâmico em que escrevemos sua entropia em
termos das variáveis extensivas $S(X)$. De maneira geral visualizamos essa
situação como as variáveis extensivas uniformemente distribuidas ao longo da
extensão do corpo, porém na realidade essas grandezes flutuam ao longo do tempo,
de forma que a princípio podemos observar uma parcela $\Delta$ em proporções
diferentes do resto do corpo $X-\Delta$:
\begin{figure}[H]
    \centering
    \begin{tikzpicture}
        \begin{scope}[xshift=-2.5cm]
            \draw[very thick,fill=lightgray]
                (0,0) -- (3,0) -- (3,3) -- (0,3) -- cycle;
            \node
                at (1.5,1.5) {$X$};
        \end{scope}
        \draw[ultra thick,->]
            (1,1.5) -- (2,1.5);
        \begin{scope}[xshift=2.5cm]
            \draw[very thick,fill=lightgray]
                (0,0) -- (3,0) -- (3,3) -- (0,3) -- cycle;
            \draw[thick,fill=gray,use Hobby shortcut,closed=true] 
                (0.7,1.5) .. (1.2,1.8) .. (1.1,2.0) .. (1.3,2.2) .. (0.7,2.3);
            \node
                at (0.85,2) {$\Delta$};
            \node
                at (2,0.8) {$X-\Delta$};
        \end{scope}
    \end{tikzpicture}
\end{figure}
Nessa situação a entropia varia de $S(X)$ para $S(\Delta)+S(X-\Delta)$, e
portanto será proibida pela Segunda Lei quando isso corresponder a uma
diminuição de entropia. Ou seja, concluimos que os pontos de equilíbrio
possíveis para nosso sistema são apenas aqueles $X$ onde
$$S(X)\geq S(X-\Delta)+S(\Delta),$$
para toda variação $\Delta$, e a entropia é uma função \emph{côncava} das
variáveis de estado extensivas. Do contrário, observariamos rapidamente
fenômenos de \emph{nucleação} como descritos acima. O caso de igualdade precisa
ser incluido devido a extensividade, e só é encontrado no caso em que $\Delta
\sim X$.

Supondo agora apenas que a temperatura termodinâmica do nosso sistema é
positiva, uma simples manipulação análoga à que foi feita na condição de
equiilíbrio nos leva também a concluir a \emph{convexidade} da energia interna
do sistema em termos das variáveis extensivas:
$$U(X)\leq U(X-\Delta)+U(\Delta).$$
Lembrando da expressão da primeira lei para processos contínuos, podemos também
escrever uma relação entre os diferenciais das variáveis de estado que precisa
ser satisfeita para todo processo possível em qualquer estado de equilíbrio
estável:
$$dTdS+dy_1dX_1+\cdots+dy_ndX_n\geq0,$$
onde a igualdade segue da relação de Gibbs-Duhem em uma transformação de escala
dada por $(dS,dX)\sim(S,X)$.

Para outros potenciais termodinâmicos, novamente nos interessamos apenas na
variação das variáveis extensivas que sobraram, e é fácil de ver que a
convexidade nestas é preservada pela transformação de Legendre.

\subsection{Fases da Matéria}

\subsection{Problemas}

\begin{enumerate}
    \item
        Com um filamento elástico foram realizados experimentos e mostrou-se que
        seu comprimento livre $L$ e constante elástica $k$ em torno da
        temperatura $T_0$ são dados por
        $$L=L_0(1+\alpha(T-T_0)),\quad k=k_0(1+\beta(T-T_0)).$$
        Além disso, mediu-se que a capacidade térmica do elástico a comprimento
        constante $C_X$ é independente de sua temperatura.
        \begin{enumerate}
            \item
                Mostre que a capacidade térmica a comprimento constante é também
                independente do comprimento $X$ do elástico
            \item
                Sendo $S_0$ a entropia do filamento no seu comprimento livre a
                temperatura $T_0$, dê a expressão para entropia do elástico como
                em termos de $X$ e $T$ e da capacidade térmica $C_X$.
                \answer{}
            \item
                Encontre a capacidade térmica a tração constante $C_f$ do
                elástico
                \answer{}
        \end{enumerate}
\end{enumerate}
